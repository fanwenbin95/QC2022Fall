\section{复习}
\courseTime{Dec 05, 11th}
\chat{%
今天到场的同学有点少啊(只有 8 位同学),看来我们要点名了。

我们还有 3 次课(算上本次),最后一次课徐昕老师来讲密度泛函,他会从头开始讲密度泛函的原理和发展。

先来回顾前面三次课讲的内容,主要是微扰和变分。本课是量子化学的基本原理和应用,跳出课程来看,化学科研的三种基本手段是实验、理论和模拟,从课程属性来看应当归到理论中,事实上任何理论想要真正地解释和指导实验,都必须要用计算机模拟。

本课讲过了量子力学基本原理和公设,也推导为数不多的可解系统,我们会发现,稍微复杂一点的体系就没有精确解,那么由此诞生了数值模拟,即通过数值近似去求解复杂系统。既然有近似就有误差,现代数值模拟可以说是精度和效率之间平衡的艺术。

借助计算机的快速发展,目前基本实现了通过数值模拟连接了实验和理论,也可以说数值模拟填平了理论和实验之间的鸿沟,让理论走在实验的前方,未来也必将实现用理论去设计实验的终极目标。所以,量子化学的大方向是基于量子力学基本原理的数值方法的开发,从这个角度来说,本课前面的所有内容都可以归纳到数值模拟的范畴,相信同学们在物化课上都已略知一二。
}

我们讲变分和微扰,根本原因是薛定谔方程不能精确求解,
\begin{align}
    \hat H \psi = E \psi,
\end{align}
这里写出的是定态、不含时、非相对论的薛定谔方程,这儿有很多约束条件。% 2022-12-05 12:47:01  Wenbin Fan @FDU
\chat{%
因为一旦其中的势函数显含时间或时间空间不能分离,那就不能得到定态薛定谔方程。幸运的是,人们感兴趣的很多化学性质,都是能由定态薛定谔方程求解出来的。诚然,一旦涉及到动力学性质等,薛定谔方程就不够用了,需要用含时薛定谔方程。话说回来,变分和微扰是求解薛定谔方程的方法,必然会考。}

变分法说的是,能量的期望值总是可以写成
\begin{align}
    \bar E = \frac{\langle \psi | \hat H |\psi\rangle}{\langle \psi | \psi \rangle} \geqslant E_0,
\end{align}
变分原理告诉我们能量大于等于 $E_0$,其中的 $\psi$ 是试探波函数,试探波函数必须满足哈密顿量的边条件。
\chat{这里包含了量子力学中两个最基本的假设。第一,微观粒子的所有行为都由波函数表述。我们如何从波函数中抽提出物理可观测量?讲该可观测量所对应的算符作用在波函数上,如果这个算符是可观测量的本征函数,那么就得到了本征值,如果不是本征函数,那么可以积分得到期望值。

变分法很有趣。如果我们将变分法当作原理,可以看出变分法给出了一条探索体系基态能量的途径。这是从物理角度来说的,如果从数学角度切入,或者从量子力学基本原理阐述,变分法将薛定谔方程这个偏微分方程,转化成了积分的方程,并找到了波函数与基态能量的映射关系,这种映射关系也可以称作泛函。

什么是泛函?我们知道函数是 $x \xrightarrow{f} y$ 数值到数值的映射,泛函是函数到数值的映射。从更朴素的角度来说,函数或泛函的映射关系,相当于以某种方式消除了自变量,在变分法中,这种消除方式是全空间积分。

用变分法求解能量的极小值,相当于寻找``试探波函数-能量''泛函上的极小值。这里为什么不说是最小值,因为人们通常不可能掌握一个高维函数的全部信息,找到极小点仅相当于井底之蛙,在井底是不可能知道旁边是否有更深的井。更严格地说,寻找全局最小值是个 NP 问题。

实践当中,通常都以优化极小点为目标,部分课题组致力于探索完整的势能面空间。只要初始猜测足够接近极小点,都是可以优化到极小点的。这里的足够接近,定义是在极小点的二次区域内,只要在这个区域内,这些利用二阶导数的优化算法均可以顺利优化到极小点。

在变分法中,这个实际的优化问题就变成了怎么设计初始猜测波函数。我们已经在在上机课中体会到了,只要我们设定好简单的参数,如坐标、电荷、自旋多重度,Gaussian 程序包自动就能算好波函数和能量。如此自动化的程序,背后蕴藏着几代科学家数十年的心血。即使你不懂什么是初猜,不懂任何数值算法,简单几步就能在计算机中模拟一个分子。现代量化软件越来越稳定,成为了黑箱。
% 很多实验组买了量化软件,简单学习软件操作后就可以迅速上手计算。

当我们用量化软件模拟得到光谱等可观测量,唯一需要考虑的是当前结果的不确定性有多大,可以在多大的置信范围内解释实验。所以,量化软件就相当于实验仪器,而且是非常稳定的仪器。
}
\suppInfo{驻点的分类}{
在应用了 BO 近似后,体系的势能由原子核坐标决定,由此产生了势能面 potential energy surface 的概念。如果体系有 $N$ 个电子,那么势能是 $3N$ 维度的函数,如果不考虑平动和转动,则势能是 $3N-6$ 维度的函数。这种高维函数通常无法直接可视化,也被称作超曲面 hyper-surface。

势能面上仅有个别区域是人们关注的,比如对应于稳定结构的极小点、对应于过渡态的鞍点,以及连接极小点和鞍点的反应路径。极小点和鞍点都是数学中一阶导数为 0 的点,且极小点的二阶导数为正,鞍点的二阶导数仅有一个负的本征值。对于函数的极小点,必须满足
\begin{align}
\pdv{E}{\lambda} = 0, \ \text{且} \ \pdv[2]{E}{\lambda} \geqslant 0. 
\end{align}
数学上极小点的判定条件,同样也可以应用在变分法中。

能量的二阶导数是力 $F = - \nabla V$,极小点要求二阶导数为正,通俗地说就是任意方向都是开口向上的函数。
}

% % 2022-12-05 08:20:03  Wenbin Fan @FDU
% 量化程序,黑箱
% uncertainty,仪器
% 如何提高精度?

变分法的困难,如何更有效地构建符合边界条件的初猜波函数 $\{\psi\}$。
\chat{波函数包含体系的所有信息,必须是 $3N$ 维度的,往后学习了自旋还要加入自旋维度,所以实际上是 $4N$ 维的。构造如此高维的空间是很困难的,当然了,搜索这个高维空间也很困难,所以我们需要线性变分法。

接下来还有更复杂的组态相互作用,逐渐逼近真实的解,但是这个逼近过程所消耗的计算资源是急剧增加的。以朴素的单电子近似方法 Hartree--Fock 为例,它的计算标度 scaling 是 $N^4$,这里的 $N$ 是基函数的个数。我们在前面推导过,\num{10000} 个水的系统就要耗时数百年,遑论标度更高的方法了。
% full-CI 是指数增长的。

这种指数增加的耗时是非常恐怖的。如果把加法比作一级增长,那么乘法是二级增长、乘方则是三级增长,近期知乎上的热门话题葛立恒数 $G(64)$ 则是第四级的超乘方,很多网友试图构造超越葛立恒数的大数,实际上连 $G(1)$ 都没有超越。

量子化学中,特别是电子结构计算中,这种指数增长被形象地称作``指数墙''。指数函数的图像的尾巴会急剧增长,这像一堵墙一样挡住了往更大体系的去路。张颖老师开发了适用于超大规模并行的 CCSD(T) 程序,在 \num{10000} 个核心上取得了超过 80\% 的效率,相当于将耗时减少了 4 个数量级,也就是耗时从 1 个月变成了 $\sim$320 秒。尽管这是个振奋人心的成就,但是在指数墙面前依然杯水车薪,哪怕是再增加一个电子,很可能这辈子都算不完了。

现代计算机工业经过了几十年的蓬勃发展,到目前为止摩尔定律依然有效,超级计算机占据了大国博弈的主战场,中美双方``你方唱罢我登场''。现阶段唾手可得的氢分子模拟,在 30 年前是难以想象的计算量。

简单举一个例子,如果要储存 10 个电子的波函数,需要消耗多少内存,再增加一个电子需要多少内存?假设每一个维度有 20 个点(或者说划分成 20 个格),每个格点消耗 \SI{1}{bytes},那么 10 个电子的波函数需要消耗 $20^{4\times 10}\,\text{bytes} = \SI{1.10e52}{bytes}$,11 个电子的波函数需要消耗 $20^{4\times 11}\,\text{bytes} = \SI{1.76e57}{bytes}$,增长了 $20^4 = \num{160000}$ 倍。
}

% % 2022-12-05 08:26:50  Wenbin Fan @FDU
% 背后原因是波函数过于复杂。

\textbf{微扰法}是一种``分而治之''的方法,因为直接求解不现实,所以分步达到目标。核心思路有两条,第一是构造与真实波函数有相同边界条件的零级哈密顿量,
\begin{align}
\hat H_0 \psi_n^{(0)} = E_n^{(0)} \psi_n^{(0)}
\end{align}
可以精确解,第二是近似哈密顿与真实哈密顿的差足够小,
\begin{align}
    \hat H' = \hat H - \hat H_0,
\end{align}
\chat{%
一口吃不成个胖子,一步也没法登上月亮,但是为了登上月亮,人们可以先爬上摩天大楼的楼顶,当然这个摩天大楼和地月距离的对比有些夸张。

第一点边界条件的要求是为了用这一组零级波函数展开真实波函数,第二点微扰足够小的要求是确保高级别微扰收敛。如果一级微扰都会发散,微扰就没有意义了。
}

有了微扰的定义,再引入\boldtext{线性变分},
\begin{align}
    \hat H(\lambda) = \hat H_0 + \lambda \hat H', \quad \lambda \in[0,1],
\end{align}
其薛定谔方程为
\begin{align}
    \hat H(\lambda) \psi_n(\lambda) = E_n(\lambda) \psi_n(\lambda),
\end{align}
可以严格地写出其波函数和能量,
\begin{align}
&\psi_n(\lambda) = \psi_n^{(0)}(\lambda) + \lambda\psi_n^{(1)}(\lambda) + \lambda^2 \psi_n^{(2)}(\lambda) + \cdots, \\
&E_n(\lambda) = E_n^{(0)}(\lambda) + \lambda E_n^{(1)}(\lambda) + \lambda^2 E_n^{(2)}(\lambda) + \cdots,
\end{align}
当 $\lambda \rightarrow 1$ 时,可得真实体系的能量。
% 2022-12-05 08:31:18  Wenbin Fan @FDU
% 2022-12-05 08:31:43  Wenbin Fan @FDU
\chat{%
用线性变分法,不再有构造边界条件的困难,因为我们已经通过零级哈密顿量构建了展开真是波函数所需的辅助基函数。
}
微扰的困难在于如何合理地构建 $\hat H_0$ 使之更接近于 $\hat H$。
\chat{%
这包含了两层目标,第一是为了微扰更快地收敛到真实解,如果 $E_n^{(0)}$ 更接近真实能量,那么就可以算更少的微扰项。第二个是更基本的问题,为了使得 $H_0$ 接近 $H$,以便在数学上展开合理,也就是 $\lambda \in [0,1]$ 不能超出波函数和能量的收敛半径。特别在强关联体系中,需要考虑,人为选定组态波函数,才能继续做微扰。

刚才用半小时梳理了变分和微扰,因为期末考试会有近似方法的大题。不用担心,所有的技术细节不会超出作业的难度。

变分和微扰都是\textbf{数值求解}薛定谔方程的方法,各有优劣。有无其它方式逼近薛定谔方程的解呢?现在我们讲一种新的方法——\textbf{密度泛函理论}。最后一次课由徐昕老师讲授密度泛函理论,预计内容会非常丰富。为了帮助理解,接下来先铺垫一些密度泛函理论的基本概念,以便同学们在最后一次课能畅快地理解。
}

\section{DFT 基础}
\chat{%
前面复习了两种基于波函数的近似方法——变分和微扰。变分法的思路是将薛定谔方程这一偏微分方程转化成波函数积分的泛函,使得能量成为波函数的泛函。
}
体系能量 $E$ 可以是体系密度 $\rho$ 的函数吗?
\chat{%
如果从波函数入手,并不能直接推知。但这个想法持续吸引着人们,因为波函数实在太复杂了,不考虑自旋已经是 $3N$ 维的。人们需要一种更简单的方式求解薛定谔方程。
}
% 波函数方法,是【?】
% 能量一定是波函数的函数,这是按照定义得到的。那么,能量可以是体系密度的函数吗?如果我们从波函数的函数来看,其实这并不能直接推导出来。为什么这个想法从一开始就这么吸引人?我们讲过波函数最大的问题是太复杂了!不考虑自选都有 $3N$ 维度的,但是
密度是空间中任意一点找到电子的概率,只有 $3N$ 维,全空间积分
\begin{align}
    \int \rho(r)\,\dd r = N. 
\end{align}
\chat{密度不同于波函数。波函数不是物理可观测量,但电子密度是可观测量,比如通过 X 射线``打出''电子分布。

从密度得到能量,这个想法非常具有吸引力。具体如何实现?
}

我们遇到的第一个问题是,什么是密度?
\chat{%
哥本哈根统计学派告诉我们,波函数模的平方是概率,
\begin{align}
|\psi(\vec r_1, \vec r_2, \cdots, \vec r_N)|^2 \,\dd \vec r_1\, \dd\vec r_2 \cdots \dd\vec r_N,
\end{align}
这个概率是否是密度?并不是。
波函数的模方指``在 $\dd r_1$ 位置找到第一个电子、在 $\dd\vec r_2$ 找到第二个电子、……在 $\dd \vec r_N$ 找到第 $N$ 个电子''的概率。所以,波函数一旦写成函数形式,每个电子在波函数中的位置是固定的、电子在波函数中的排序是隐含的变量。
% 就是变量可以写成 r1 r2 rn,也可以互换,本身变量的位置隐含着【】,顺序是不会变的,
% 电子是可区分的,第一个变量对应第一个电子,这是函数中约定的概念。
这个还是蛮复杂的,Igor 当年就搞混了。
% 波函数的模方指的是 $N$ 个电子分别出现在 r1 r2 ... 的概率。
}
由于粒子的不可区分性,精确的多体波函数必须满足
\begin{align}
|\psi(\vec r_1, \vec r_2, \cdots, \vec r_i, \vec r_j, \cdots, \vec r_n)|^2 = |\psi(\vec r_1, \vec r_2, \cdots, \vec r_j, \vec r_i, \cdots, \vec r_n)|^2
\end{align}
\chat{%
因为电子的不可区分性,``第 $i$ 个电子出现在 $\vec r_i$、第 $j$ 个电子出现在 $\vec r_j$'' 的概率,必须等于``第 $i$ 个电子出现在 $\vec r_j$、第 $j$ 个电子出现在 $\vec r_i$'' 的概率。显式地写出波函数的自变量,必然会人为指定每个电子的顺序,这是数学记号的缺陷。

单个电子的密度,可以用谐振子或一维势箱辅助理解。对于多个电子,尽管我们能够在波函数中标记每个电子,但仍要提醒自己电子是不可区分的。
}
% 有啥问题吗?哥本哈根的统计诠释,波函数模的平方对应于概率密度。一个电子的概率密度,如可精确求解的谐振子,可以画出来一维的概率密度。对于多个电子,我们可以标记每个电子,又要强制告诉自己电子是不可区分的,所以我们必须要用到不可区分性质。

% 我们写出一个函数 $\psi(1,2,3,4,5,6)$,那么 1 是第一个变量,对应第一个电子的坐标,这是函数的定义。我们又要粒子不可区分,两个电子交换实空间的坐标,那么概率也要相等。

\courseTime{2 of 4}
\chat{尽管本课是选修课,我相信选了这门课的同学都是对量子化学感兴趣的。今天很多同学不来,非常遗憾不能带领那些同学在量子化学的大海里遨游了。

本节课的密度泛函不在考试范围,接下来的自旋、泡利不相容原理、Hartree--Fock 会考到的,试题难度不会超过课后习题。
}

从全同粒子的\textbf{不可区分性}出发,给定一组空间点 $r = (\vec r_1, \vec r_2, \cdots, \vec r_N)$,任意交换两个电子,共有 $N!$ 种排布方式,那么密度
\begin{align}
    \rho(\vec r_1, \vec r_2, \cdots, \vec r_N) = N! \,|\psi(\vec r_1, \vec r_2, \cdots, \vec r_N)|^2,
\end{align}
\chat{%
电子是全同粒子,第一个位置可以是 1、2、3、$\cdots$、$N$,第二个位置可以是 2、3、$\cdots$、$N$,依次类推,可得到 $N(N-1)(N-2)\times\cdots2\times1 = N!$ 种互换方式。

这里的密度是指,在这一组空间点 $r$ 处找到任意一个电子的概率,我们并不关心是哪个电子。系数 $N!$ 确保了任意交换两电子的坐标后概率都不变,因此我们从概率得到了密度。

从不可区分性可以看出,单电子的试探波函数比较容易构造,多电子的试探波函数就很难构造了。
% 这里的 $r$ 表示一系列坐标,这种记法称作 compact notation。%% 搜了一下并没有这种叫法,应该是跟电子态里的什么记法搞混了,参考 Charge and Energy Transfer Dynamics in Molecular Systems 2011
}
% % 2022-12-05 09:01:46  Wenbin Fan @FDU
% 如果波函数不满足这个条件,那么满足这个规则就很困难【】,
% 所以这里不区分电子【】。

% 2022-12-05 09:03:23  Wenbin Fan @FDU
我们再做一个事儿,单独考虑每个电子的概率密度,
\begin{align}
    \rho_1 (\vec r_1) &= \int \cdots \int |\psi(\vec r_1, \vec r_2, \cdots, \vec r_N)|^2 \, \dd\vec r_2\, \dd\vec r_3 \cdots \dd\vec r_N, \\
    \rho_2 (\vec r_2) &= \int \cdots \int |\psi(\vec r_1, \vec r_2, \cdots, \vec r_N)|^2 \, \dd\vec r_1\, \dd\vec r_3 \cdots \dd\vec r_N, \\
    &\vdots \notag
\end{align}
\chat{$\rho_1(\vec r_1)$ 表示第一个电子出现在 $\vec r_1$ 的密度,其它电子的坐标全都积掉,同理 $\rho_2(\vec r_2)$ 表示第二个电子在 $\vec r_2$ 的密度。}
因为全同粒子的不可区分性,各个电子在 $r_1$ 处密度都是相等的
\begin{align}
\rho_1(\vec r_1) = \rho_2(\vec r_1) = \cdots = \rho_N(\vec r_1),
\end{align}
因此,整个体系的密度可以写成
\begin{align}
    \rho(r) = N \rho_1(\vec r_1),
\end{align}
\chat{不用考虑是哪个电子在 $\vec r_1$ 处形成的密度,同样由于不可区分性,总密度就是单个电子密度的 $N$ 倍。}
% 【】定义了密度。对于 $\rho_1$ 的积分,怎么积出来?
% 【】全空间积分为 1 的,那么所以
对于第一个电子的密度 $\rho_1$,积分为
\begin{align}
    \int \rho_1 (\vec r_1) \dd\vec r_1 = 1,
\end{align}
\chat{%
$\rho_1$ 已经积掉了 $\vec r_2$、$\vec r_3$、$\cdots$、$\vec r_N$,剩下 $\vec r_1$ 积分相当于要补齐波函数的归一化,是要等于 1 的。这里有些类似循环论证,因为我们还不能理解技术细节,只需要知道原理上是可行的。
}
因此,全空间的密度积分为
\begin{align}
    \int\rho(r) \,\dd r = N,
\end{align}
推知
\begin{align}
\rho(r) = N \int\!\cdots\!\int |\psi(\vec r_1, \vec r_2, \cdots, \vec r_N)|^2 \, \dd\vec r_1\, \dd\vec r_2 \cdots \dd\vec r_N. 
\end{align}
\chat{%
我们用一节课的时间理了密度和波函数的关系,二者最大的差别是粒子的可区分性。在波函数中,粒子被编号,由于函数的特性而必须显式写出粒子的顺序,这强制粒子成为可区分的。在密度表述中,将概率密度乘以特定权重以得到密度的概念,进而推知密度的全空间积分正是电子数 $N$。
}
% 我们用了一节课讲清楚了密度和波函数的概念。【】
% % 2022-12-05 09:08:39  Wenbin Fan @FDU
% 粒子的不可区分性,一定成为了波函数的性质,而不是从函数【】体现出来。
% 推出来密度
% \begin{align}
%     \rho (r) = N \int \cdots \int ...
% \end{align}

$\rho(\vec r)$ 仅为三维坐标的函数,那么能否直接从密度 $\rho(\vec r)$ 得到能量 $E$?

\chat{%
函数与泛函的区别在于输入不同。函数是 $x \xrightarrow{f} y$ 数值到数值的映射,一个 $x$ 只有一个 $y$,泛函是 $f(x) \xrightarrow{F} y$,函数到数值的映射。如图 \ref{fig:function_vs_fuctional}。
}
% 2022-12-05 09:11:52  Wenbin Fan @FDU
哈密顿量中,哪些是泛函?
\begin{align}
\hat H = \sum_i^N \hat T_i + \sum_{i<j}^N V_{\text{ee}}(r_{ij}) + \sum_i^N V_{\text{ext}}(\vec r_i), \quad 
V_{\text{ext}}(\vec r_i) = -\sum_A \frac{Z_A}{r_{iA}}
\end{align}
\chat{%
哈密顿量包含了动能、电子电子相互作用,以及在无外场情况下被视作\boldtext{外势}的电子-原子核相互作用,其中 $A$ 是原子核的指标、$N$ 是总电子数。原子核与电子是吸引的,所以外势项是负的。
% 没有考虑外场时,外势是电子与核的相互作用。单电子算符【】。

动能和电子-电子相互作用,二者在任何体系中都是一样的,都有通用的方法求解,比如动能部分可利用单电子近似求解。因此,二者合并起来称作\boldtext{普适泛函} $F[\rho]$。

由于原子核各不相同,外势并没有通用的求解方法。
}
单独考虑外势的期望
\begin{align}
\langle V_{\text{ext}} \rangle &= \sum_i^N \int\!\cdots\!\int |\psi(\vec r_1, \vec r_2, \cdots, \vec r_N)|^2 V_{\text{ext}}(\vec r_i) \dd \vec r_1 \dd \vec r_2 \notag \\
&=\sum_i^N \int V_{\text{ext}}(\vec r_i)\dd r_i 
\underbrace{
    \int\!\cdots\!\int |\psi(\vec r_1, \vec r_2, \cdots, \vec r_N)|^2 \dd\vec r_1 \cdots \dd\vec r_{i-1} \, \dd \vec r_{i+1} \cdots \dd\vec r_N
}_{\rho_i(\vec r_i)}
\notag \\
&= N \int V_{\text{ext}}(\vec r_i) \,\dd r_i \ \rho_i (\vec r_i) \notag \\
&= \int V_{\text{ext}}(\vec r_i) \,\dd r_i \ \rho (\vec r_i), 
\end{align}
% 【讨论 psi ** 2】
% 把第 i 个电子的【】提出来,积分掉非 i 的【】,【红色行】第 i 个电子出现在 i 的密度 $\rho_i(\vec r_i)$,
% \begin{align}
%     &=\\
%     &=
% \end{align}
% 电子是一样的,但是原子核会有不同,
% \begin{align}
%     =N \int V_{ext} (\vec r_1) []
% \end{align}
所以外势能是密度\textbf{显式}的泛函。
\chat{%
这里利用了电子是全同粒子的这一等价性,某点的密度不会因为是第一个电子就吸引一些、最后一个电子就排斥一些,因此得到了系数 $N$。

% 为什么要给大家讲这个?明天要给华中师范大学的学生讲四节课,准备了 PPT。
徐老师的 DFT 课上应该不会讲这些基础的知识。如果你以后要做理论模拟,99.9\% 的可能性会用到密度泛函。

DFT 出发点是``能量成密度的泛函'',这个假设一定是对的吗?徐老师会部分否定这个结论,请听课堂上的讲解。
}

% \textbf{泛函}。
以\boldtext{局域密度近似} local density approximation 泛函为例,形式为
\begin{align}
E_{\text{X}}^{\text{LDA}}[\rho] = -\frac34 \left(\frac34\right)^{\frac13} \int\dd\vec r [\rho(\vec r)]^{\frac43}.
\end{align}
\chat{%
这是一种交换相关泛函,密度指数上的 $\frac43$ 是均匀电子气模型推导来的。我们不关心它是如何构造的,先对 DFT 中的泛函有个直观的了解。

% 不用管这是啥,以后交换相关用得到。LDA 的泛函
% 也不用管它是怎么来的。
从公式里能清楚地看到,能量是密度的函数。
}

% 【】
% \boldtext{泛函的偏导}。对于函数
能量是泛函的最小值,必然会涉及到微分。
\suppInfo{泛函的微分}{
微分是指输出对输入微小变化的响应,
\begin{alignat}{2}
&\delta f = f(x+\delta x) - f(x) &&= \left(\frac{\delta f}{\delta x}\right)\delta x 
%= f' \cdot \delta x
, \\
&\delta F = F[f + \delta \!f] - F[f] &&= \int \dd x \left(\frac{\delta F}{\delta f}\right) \delta \!f,
\end{alignat}
这里不再展开数学讨论。函数或者泛函,可以理解为消去自变量的一种方式。通常来说,可微分的函数对数值算法比较友好,
% 最好的方式是全空间积分,
% 通常数值可导可微的是积分,
由此定义积分形式的泛函为\textbf{最简泛函},
\begin{align}
    F[y(x)] = \int F(x, y, y')\,\dd x. 
\end{align}
}
外势的期望值是密度的泛函 $V_{\text{ext}}[\rho]$,其微分为
\begin{align}
\delta V_{\text{ext}}[\rho] 
&= \int \dd \vec r \, V_{\text{ext}}(\vec r) 
\bigl[
    \rho(\vec r) + \delta\rho(\vec r)
\bigr] - 
\int \dd \vec r \, V_{\text{ext}}(\vec r)\, \rho(\vec r) \\
&= \int \dd \vec r \, V_{\text{ext}} \, \delta \rho(\vec r) 
% not break line % 2022-12-06 17:28:16 Wenbin FAN @FDU
= \int \dd\vec r\, \frac{\delta V_{\text{ext}}[\rho]}{\delta\rho} \delta\rho, 
\end{align}
对于 LDA 泛函,令其系数为 $A_{\text X} = - \frac34\left(\frac3\pi\right)^{\frac13}$,其微分为
% \begin{align}
% \langle V_{\text{ext}} \rangle = \int \dd r V_{ext} (\vec r) (\rho(r) + \delta \rho(r) ) - \int \dd r V_{ext} (\vec r) \rho(\vec r) [][][]
% \end{align}
\begin{align}
\delta E_{\text{X}}^{\text{LDA}}[\rho] 
&= A_{\text X} \left\{
    \int \dd\vec r \bigl[
        \rho(\vec r) + \delta \rho(\vec r)
    \bigr]^{\frac43} - \int \dd\vec r\, [\rho(\vec r)]^{4/3}
\right\} \notag\\
& = A_{\text X} \left\{
    \int \dd\vec r\left[
        [\rho(r)]^{\frac43} + \frac43 [\rho(r)]^{\frac13} \,\delta \rho(\vec r) + \cdots
    \right] - \int \dd\vec r\, [\rho(\vec r)]^{\frac43}
\right\} \notag\\
&= A_{\text X} \int\dd\vec r \ \frac43 [\rho(\vec r)]^{\frac43} \,\delta \rho(\vec r),
\end{align}
泛函微分之后依然是函数。
% 偏微分给出了一个 kernel 核,微分完是个函数。有了这些定义,可以来操作【】。
\chat{
上式中非整数次幂,同样可由二项式定理展开,
\begin{align}
(x+y)^\alpha = \sum_{k=0}^{\infty}C_\alpha^k x^{\alpha - k}y^k, 
\quad C_\alpha^k = 
{\alpha \choose k}
 = \frac{\alpha (\alpha-1) \cdots (\alpha-k+1)}{k!}. 
\end{align}
}

% 有了这些基本定义,就可以来操作

哈密顿量中,系统依赖的参数有电子数 $N$、核电荷数 $Z$、原子核坐标 $\vec R_{\text{A}}$。由薛定谔方程,原则上可以给出精确的波函数,进而得到任何可观测量。人们提出了许多方法来近似求解,比如本课讲过的变分和微扰。
\chat{%
动能算符 $\hat T$ 和电子-电子排斥势 $V_{\text{ee}}$ 仅由电子数 $N$ 决定,不管系统如何变化,这两者是不变的,所以称作普适算符。
}
如果 $\rho_0(r)$ 唯一映射 $(N, Z, R_{\text{A}})$,那么密度可以确定能量。所以问题来啦,
\homework{\textbf{11.1} 不用反证法,由密度 $\rho(\vec r)$ 直接得到 $\hat H$,包括总电子数 $N$、总原子核数 $M$、核电荷数 $Z_{\text{A}}$、原子核坐标 $\vec r_{\text{A}}$。

提示 (\textbf{1}) $R_A$ 由密度的尖点决定,(\textbf{2}) $\lim_{r_{i、text{A}}\rightarrow 0} \qty[\pdv{r} + 2Z_\text{A}] \rho(r) = 0$。}
\chat{%
本题的结论可以从侧面证明,密度泛函一定存在,尽管表达形式不确定。

% 试卷中会有一点点密度泛函的内容。
}

\courseTime{3 of 4}
%%%% 笔者的电脑出问题了,这节课没用电脑记笔记,回去听录音整理吧,质量恐怕会下降(当然本来就不太高(
% 2022-12-05 14:13:58  Wenbin Fan @FDU
自旋存在的证据
1. 钠原子 黄光(D线)
有两条间隔很近的谱线
2. 1925年,Uhlenbeck和Goudsmit基于原子光谱精细结构提出了电子除了轨道角动量,还有自旋角
动量。
在Dirac的相对论量子力学中,电子自旋自然出现。对相对论量子力学做非相对论演化,可以得到电子自
旋在非相对论情况下的算符表述


% 2022-12-05 14:27:38  Wenbin Fan @FDU
\courseTime{4 of 4, Dec 05}
利用规则,给出递增递减后新的本征函数,并且这些本征函数都具有相同的性质:Sz 不同的本征值、S2 相同的本征值。

接下来的问题是,对 $k$ 有什么限制条件?定义
\begin{align}
Y_k = \hat S_{\pm}, []
\end{align}
作用两次有
\begin{align}
    \hat S_z^2 Y_k = [].
\end{align}
将 {15} 减去 {17} 可得
\begin{align}
    [][]
\end{align}
代入 {1} 可得
\begin{align}
    (\hat S_x^2 + \hat S_y^2) Y_k = [c - (b \pm k\hbar)^2] Y_k, 
\end{align}
左侧显然是非负的物理量,那么可以推知右侧的本征值为正 $c-(b\pm k\hbar)^2 \geqslant 0$,移项开方后
\begin{align}
    -\sqrt c \leqslant b \pm k\hbar \leqslant \sqrt c,
\end{align}
因此我们得到了 $b \pm k\hbar$ 的上下限,设上下限分别为 $b_{\text{max}}$、$b_{\text{min}}$,对应于 $b_k$ 的最大值和最小值。把上下限代回 $\hat S_z$ 中有
\begin{align}
    \hat S_z Y_{\text{max}} = b_{\text{max}} Y_{\text{max}}, \quad \hat S_z [][]
\end{align}

% 2022-12-05 14:35:36  Wenbin Fan @FDU
把 z 两次作用上去得到平方,又因为 S2 对应于 c,那么递增递减所构造的本征函数【?】。
递增递减不改变 c,

将递增算符作用上去,
\begin{align}
    [][]
\end{align}
% 2022-12-05 14:38:26  Wenbin Fan @FDU
大于最大值,不成立
【少拍了一行】

同理,将递减算符作用上去,得到
\begin{align}
    \hat S_- \hat S_+ Y_{\text{max}} = 0. 
\end{align}
利用 {7} 式,
\begin{align}
    (\hat S^2 - \hat S_z^2 - \hbar \hat S_z) Y_{\text{max}} &= 0 \\
    c - b_{\text{max}} ^2 - \hbar b_{\text{max}} &= 0\\
    c &= b_{\text{max}} ( b_{\text{max}} + \hbar),
\end{align}
同样的论证过程可以证明,由 $\hat S_- Y_{\text{min}} = 0$ 可推知
\begin{align}
    c = b_{\text{min}} ( b_{\text{min}} - \hbar), 
\end{align}
联立 {20} 和 {21} 可得
\begin{align}
    b_{\text{max}}^2 + \hbar b_{\text{max}} - b_{\text{min}}^2 + \hbar b_{\text{min}} = 0,
\end{align}
解得 $b_{\text{min}} = - b_{\text{max}}$,舍去另一个相等的解。又因为
\begin{align}
    b_{\text{max}} - b_{\text{min}} = n\hbar, 
\end{align}
立刻知道
\begin{align}
    b_{\text{max}} = \frac12 n\hbar, \quad n = 0, \frac12, 1, \frac32, \cdots,
\end{align}
【】【】

% 2022-12-05 14:48:20  Wenbin Fan @FDU
如何定义 k,如何确定量子化条件。这里没问题就往下推了。这对同学们应该都是新知识,【】

% 2022-12-05 14:49:15  Wenbin Fan @FDU
利用自旋算符的互易关系,在非相对论的近似框架下的近似,一到相对论的框架下就不是这种形式,会有自旋-轨道耦合,那么自旋就不是一个\boldtext{好量子数}了。这有点类似 BO 近似下的电子和原子核运动可分离。那么在相对论近似下,我们只给出了【】。如何求出了量子化条件?由递增递减算符,看到了新的本征函数,并且与原来的本征函数有量子化的跃迁。升降算符不是凭空的来的,【】,两个作用,第一是 S2 Sz 作用上去有新的本征函数,【】,可以导出 $k$ 的限制,进而得到量子化的条件。

自由粒子的波动方程,你可以发现在解无限深势阱的时候,也是这个方程,不过我们需要约束边界条件。目前自旋部分我们还有没有做任何的约束,只要满足 {1} 的关系,本征函数都可以写成上面的形式。电子也有类似的关系【?】,角动量是整数,这与自旋不一样。【】

% 2022-12-05 14:55:06  Wenbin Fan @FDU
这个量子化条件过于宽松。2 是 d 轨道,$m_l$ -2, -1,0,1,2,我们约束了 $\hat L$ 的本征值为整数,否则多项式不能截断。

因为本课不涉及更高级的理论,所以这里不加证明地引入实验结论:\textbf{所有电子具有单一的 $S$ 数值},中子与之子的自旋也是 $\frac12$,$\pi$ 介子。
【录像】
% 2022-12-05 14:58:55  Wenbin Fan @FDU
[]
自旋给出了量子化条件是普适的,【】只能看到两条谱线,$S = \frac12$, $m_s = -\frac12, \frac12$,

% 2022-12-05 14:59:58  Wenbin Fan @FDU
\begin{align}
    \hat S^2 Y = \left[\frac12\left(\frac12 + 1\right)\hbar^2\right] Y  = \frac34 \hbar^2 Y,
\end{align}
那么 $|S| = \frac12 \sqrt3 \hbar$,【图】
也就是说,电子自旋只有两个本征态。这与角量子数很像。一旦定了自旋是 S2 Sz 的本征函数,那么就一定不是 Sx Sy 的本征函数,那么 Sx Sy 与 Sz 也不对易。

% 2022-12-05 15:02:57  Wenbin Fan @FDU
【】
没有用到实空间的表述。所以把自旋当成【】,定义
\begin{align}
    \alpha(m_s) = \begin{cases}
        1, \quad & m_s = \frac12, \\
        0, \quad & m_s = -\frac12, 
    \end{cases}, \quad 
    \beta(m_s) = \begin{cases}
        0, \quad & m_s = \frac12, \\
        1, \quad & m_s = -\frac12, 
    \end{cases},  
\end{align}
全空间积分
\begin{align}
\langle \alpha | \beta \rangle = \sum_{m_s = -\frac12}^{\frac12} \partial ^* [][]
\end{align}
有了这个定义之后,单电子的波函数必须拓展到自旋,
\begin{align}
    [] \rightarrow
\end{align}
归一化条件也要扩展成
\begin{align}
    [][]
\end{align}
为什么自旋是量子力学公设?因为是从实验总结出来的公设,不能简单地从非相对论量子力学原理中导出,我们要把自旋当成粒子内禀的性质。

有了这样的说法,
自旋与氢原子,
\begin{align}
    []
\end{align}
哈密顿算符跟自旋算符一定是互易的,因为二者没有关系,所以总是可以把氢原子的薛定谔方程写成
\begin{align}
    []
\end{align}
氢原子的简并度 $n^2 \rightarrow 2n^2$,所有的谱项都会一份为二。自从实验上观测到谱线的精细结构,(压力给到理论这边)

