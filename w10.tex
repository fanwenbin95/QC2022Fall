\section{复习}
\courseTime{Nov. 28, 2022. 10th, 1 of 4}

上周上机课的人最多。大家对这门课的感觉是什么?

今年是这门课的第三年,任课老师们一直在探索如何授课。目前来说,一直在推导基础知识,讲明白为止。

本课的内容是量化的原理和应用,理应更接近化学中的实际问题,但大部分内容是数学物理推导,远没有达到现阶段的量化。接下来两周的自旋、全同粒子,之后才是真正涉及到量化的基本算法。

先前的物化 1 已经给本课做了很好的铺垫。等各位到了研究生阶段,如果做理论或计算,还要学习量化。

我相信很多同学对量子力学一直是很懵懂,究其原因是微观粒子的和宏观物体的行为是完全不同的,如果利用直觉去理解量子力学是基本不可能的,只能借助数学工具。物理学家费曼言,``没有人能理解量子力学''。随着学习的深入,物理越来越像哲学,我们必须改变思维方式,以适应全新的物理图形,量化也是这样。

我们到目前为止,只是做了些准备工作。目的是,在日后的科研中碰到相关的知识,能够回想起来课上学过相关内容,就算不理解,也相信有途径去解决。前人已经总结了很多工具,去驯服复杂的微观粒子。

在力学实验中,沿着最速降线轨道的小球最先抵达终点,这是我们看得见、摸得着的实验。原子尺度的微观世界不可能看到,因为尺度远小于可见光波长,光学显微镜失效了。借助电子显微镜,分辨率可提升三个数量级,达到 $\SI{1E-10}{\metre} = \SI{0.1}{\nano\metre} = \SI{1}{\angstrom}$,从而``看到''原子世界。同样地,有方法研究宏观系统,也会有方法研究微观系统,总体来说是先易后难、螺旋上升的,这是唯物辩证法的规律之一。

硕士的量化课还有一学期,基本上是沿着本课的足迹,拓展到现阶段使用的量子化学方法。如果你不学本课,直接去听硕士的量化课,可能会感到深深的不安,因为所有的推导都是建立在本课的量子力学基础上的。当你熟练掌握量子力学、高等量子力学、数学物理方法等工具,一定会加深对量化的理解。

% 下一周涉及到量化。

% 2022-11-28 08:09:05  Wenbin Fan @FDU
上周是微扰。
从课程设置的角度来说,为什么要讲微扰?从课程设置的角度来说,并没有期望同学们可以理解,这与高中的``深挖坑''是不同的,高考要求我们熟练应用基础知识、技巧,本课的内容囫囵吞枣地理解即可。这并不是说忽略细节,而是要有更大的框架、更高的视角俯瞰本课的架构,当你遇到类似的问题之后要能反应过来用哪种方法求解。

% 希望囫囵吞枣地理解。
微扰的原理是什么?回到
第一节课,量子粒子的行为由量子力学基本原理保证,一次量子化中的基本原理是薛定谔方程。化学中的稳态是定态薛定谔方程,定态是不含时体系分离坐标和时间后的坐标部分。微观体系的性质全部由波函数决定。波函数是 $3N_{\text{atom}}$ 和时间的复函数,由薛定谔方程解出。

同学们相信牛顿力学,可能并不相信由理论演绎出的薛定谔方程。不管相信与否,微观的量子力学理论是经得起实验验证的,它是与宏观世界完全不同的。薛定谔为什么能提出 $\hat H \Psi = E \Psi$,因为他是当年的数学物理天才,是人们心中既定的诺奖获得者。当然,薛定谔方程不是凭空出来的,是联系了物理图像和波动方程得到的。当薛定谔的波动派遇上哥本哈根的矩阵派,双方并不能相互说服,直到狄拉克用数学推导证明了二者的等价性,才统一了量子力学的两大学派。(助教推荐上海市科技馆)

量子力学的创立,体现了人类超越造物主的强悍智慧。
% 后来爱因斯坦提出相对论 % 这个与课程不相关就不写了
当理论工具足够强大,是有可能超越现在的实验,去拓宽人类知识的边缘、打破科技前进的壁垒。




% 2022-11-28 08:15:03  Wenbin Fan @FDU
科学的二元论是实验和理论。

数学是人类的最大智慧。

实验和理论

% 2022-11-28 08:24:03  Wenbin Fan @FDU


% 2022-11-28 08:29:15  Wenbin Fan @FDU
阴离子束缚态


% 2022-11-28 08:33:55  Wenbin Fan @FDU
体系波函数最好由完备基展开,那么 Gaussian 型基函数是完备的吗?但 GTO 描述不了解离行为。特别是考虑高阶激发的时候,波函数行为像平面波,这是 GTO 无法描述的。
【】完备的,角动量,指标展开得到完备基。多个原子呢?需要用原子轨道的线性组合。两个原子上的原子轨道相互之间一定不会正交,成键轨道就是两个原子上的原子轨道相互之间重叠很大。
GTO 并不是完备基。

构造基组是个非常大的学问。

% 2022-11-28 08:40:07  Wenbin Fan @FDU
当两个电子靠近时,波函数有尖峰。现在所有的基组,都无法描述这个尖峰。轨道之间是相乘的,线性组合的原子轨道构造出的波函数,其中不会显式地出现 $1/r_{12}$。

F12-MP2。基组增大,导致数值上收敛困难。所以另有方法引入了距离,如 MP2-F12、CCSD(T)-F12。

有没有完备的呢?VASP 中的平面波基组就是完备的。注意到 VASP 要求输入截断能【】。
平面波是全空间等价的。
【】
20 个、2000 个。如何处理内核电子?这就是赝势 pseudo-potential,内层电子的势能是假的。
超大稠密矩阵的对角化。

有没有既又?小波基组,【?】混合基组。这有很多方向。

从科学研究范式来说,1500 年前的炼金术是实验科学,牛顿等人构建起的经典物理是理论科学,【】是模拟科学。从 1950 年开始,计算机发展,现在计算化学已经成为了独立的学科。

% 2022-11-28 08:49:59  Wenbin Fan @FDU
掌控大自然。数学工具,计算工具,基于各种工具的研究范式。


\courseTime{Nov 28, 10th, 2 of 4}
继续上课。

微扰相对于变分。

薛定谔方程不能精确求解。变分表示?

张益唐:原来是大海捞针,现在是在大海的子集中找到一根针,后人可不断扩大这个子集。

变分法,不断扩大【】

微扰法,能否有个跳板?假设有个体系
\begin{align}
    \hat H_0 \psi_0^{(0)} = []
\end{align}
可以精确求解,同时 $\hat H - \hat H_0 = \hat H'$ 是微小差别,那么可微扰求解。构造微扰途径,线性的是最常用的,
\begin{align}
    \hat H(\lambda) = \hat H_0 + \lambda H',
\end{align}
【】
\begin{align}
    \hat H(\lambda) \psi_n(\lambda) = []
\end{align}
对波函数展开
\begin{align}
    \psi_n(\lambda) = \psi_n^{(0)} []
    E = 
\end{align}
其中 【】 是$k$ 级校正波函数,【】能量。

非简并的微扰理论,
\begin{align}
    &E_n^{(1)} = \langle \psi_n^{(0)} | \hat H' | \psi_n^{(0)} \rangle = H_{nn}', \quad H_{ij} = [], \\
    &\psi_n^{(1)} = \sum_{j\neq n}
\end{align}

% 同学问题,为什么微扰波函数正交?
% 拉格朗日展开

氦原子的电子基态的微扰,
\begin{align}
    \hat H_0 = - \frac{\hbar^2}{2m_{\mathrm e}} (\hat \nabla_1^2 + \hat \nabla_2^2) - \frac{2e^2}{r_1} - \frac{2e^2}{r_2}, \quad \hat H' = \frac{e^2}{r_{12}},
\end{align}
波函数和能量
\begin{align}
    \psi_1^{(0)} = []\\
    &E_1^{(1)} = [][] % 没拍
\end{align}
则
\begin{align}
    E_1 \approx = 
    &\psi_1 \approx = 
\end{align}

% 2022-11-28 09:19:18  Wenbin Fan @FDU
【图】
n=2 不止这两个轨道是简并的。

现在我们要问个问题。1s-2s、1s-2p 的屏蔽效应不一样,那么还能简并吗?如何求解氦原子的第一激发态?

简并能级的微扰法。需要重新讲,还是直接给答案。(同学:直接给吧,时间不太够)(笑)

8 重简并。
\begin{align}
    \psi_1^{(0)} = 1s(1) \ 2s(2), \quad & \psi_5^{(0)} = 1s(2) \ 2p_y(2), \\
    [][]
\end{align}
其中
\begin{align}
    \psi_8^{(0)} = \frac1{(4\pi)^{1/2}} \left(\frac Z{a_0}\right)^{5/2} r_1 \ee^{-\frac{Z r}{2a_0}} \cos\theta_1 [][]
\end{align}
【看文献也是】。简并的波函数描述成
\begin{align}
    \psi_n^{(1)} = []
\end{align}
如何求 $c_i$,对应求解久期方程,
\begin{align}
    \left|\begin{matrix}
        H'_{11} - E_1^{(1)} & H_{12}'
    \end{matrix}\right| = 0
\end{align}
需要求解其中的 $H_{ij}'$,公式在前面已经给出来了。

1) 容易证明
\begin{align}
    \langle \psi_i^{(0)} | \psi_j^{(0)} \rangle = \delta_{ij}, \quad i,j = 1, 2, \cdots, 8,
\end{align}
2) 
\begin{align}
    H_{ij} = []
\end{align}
利用宇称
\begin{align}
    H'_{13} = \left\langle 1s(1) \, 2s(2) \mid | [] 
    \right\rangle
\end{align}
对 $x$ 反演,

分解成了 4 个小矩阵。


% 3 of 4
上午接着讲了微扰:简并微扰的应用。

我们现在的目的是求 He 原子的第一激发态。采用无相互作用的波函数作为未微扰波函数。

[]

% 2022-11-28 13:35:02  Wenbin Fan @FDU
分块求解
\begin{align}
    [matrix]
\end{align}
其中
\begin{align}
    &H'_{11} = \left\langle 1s(1) \, 2s(2) \mid | \frac{e^2}{r_{12}} \mid | 1s(1) \, 2s(2) \right\rangle \\
    &1 []
\end{align}
同理会有
\begin{align}
    H'_{11} = H'_{22}, \quad [[][]]
\end{align}
其中
\begin{align}
    H_{11}' = [][]
\end{align}
\begin{align}
    H'_{11} = J_{1s \,2s}
\end{align}
[][][]
% 2022-11-28 13:45:42  Wenbin Fan @FDU
对角元都相等。全空间积分,先后顺序无关。
\begin{align}
    \left|\begin{matrix}
        J_{1s\,2s} - E^{(1)} & K_{1s\,2s} \\
        K_{1s\,2s} & J_{1s\, 2s} - E^{(1)}
    \end{matrix}\right| = 0
\end{align}
按照两种积分的定义,显然有
\begin{align}
    K_{mn} = K_{nm}, ][[]]
\end{align}
[]
% 2022-11-28 13:49:31  Wenbin Fan @FDU
已经分解开了【见照片】的两种简并。

那么 $c$ 怎么求?【这个 c 是前面定义处的展开系数】

% 2022-11-28 13:53:40  Wenbin Fan @FDU
一级微扰之后的波函数
\begin{align}
    &\psi_1 = \frac 1{\sqrt 2} [ 1s(1) \ 2s(2) - 2s(1) \ 1s(2) ], \\
    &\psi_2 = []
\end{align}

后面六重简并可以分成 3 种、3 种。

% 2022-11-28 13:57:17  Wenbin Fan @FDU
对应的波函数可以写成 % 哪种波函数?
\begin{align}
    & \Psi_3^{(0)} = \sqrt1{\sqrt 2} \left[
        \psi_{1s}(1) \psi_{2px}(2) - \psi_{2px}(1) \psi_{1s}(2) 
    \right], \\
    & []
\end{align}
已经求出了简并的一级微扰波函数。

我们用了电子间的库伦排斥
\begin{align}
    \hat H' = \frac{e^2}{r_{12}}
\end{align}
部分消除了简并。

那么这个积分怎么算?上节课算了非简并的积分项,这里不再求了,直接给结果,
\begin{align}
    J_{1s\,2s} = \frac{34}{81} \frac{\hbar^2}{m_{\mathrm e}a_0^2} = \SI{11.42}{\electronvolt}, \\
    H_{1s\,2p} = \frac{118}{243} \frac{\hbar^2}{m_{\mathrm e}a_0^2} = []
\end{align}

% 2022-11-28 14:02:42  Wenbin Fan @FDU
库伦积分总是大于交换积分一个数量级以上,这是个非常 general 的结论。

\homework{\textbf{10.1} 求 $J_{1s\,1s}$ 积分。}

% 2022-11-28 14:03:54  Wenbin Fan @FDU
从谱项的角度来说,能级发生了劈裂。原来是八重简并,微扰之后能量不一定下降,【图,好大一张图】

% 2022-11-28 14:07:27  Wenbin Fan @FDU
为什么库伦作用让能量上升?因为零级波函数没有相互作用。

% 2022-11-28 14:10:21  Wenbin Fan @FDU
【】劈裂出来的两个【?】,一定比 1s 2p 低

(1) 1s 2s 与 1s 2p 的能级存在错误的交叉,这是由于忽略的高阶项的激发导致的。采用变分微扰的方式,Knight\footnote{Knight 的 k 不发音}、Scherr 等人计算了二级、三级能量校正,【图】参考文献 R.E. Knight, C.W. Scherr, Rev. Mod. Phys. 35 [][][]

% 2022-11-28 14:14:58  Wenbin Fan @FDU
考虑到库仑相互作用,能级一定是要分开的。

% 2022-11-28 14:26:01  Wenbin Fan @FDU
\courseTime{4 of 4}
【】通过库仑相互作用,在一级相互作用下,就可以正确地将 8 重简并分解成 4 个简并态,1s 2p 有两个三重简并。对这个结论的第一个思考是,通过一级微扰,可以分裂能级,但一级不够。幸运的是,这个微扰比较小,还可以用微扰法。

\textbf{2. 库伦简并的消除:}
未微扰态存在两种简并,
\begin{align}
    &\text{$n$ 相同但 $l$ 不同}\quad []\\
\end{align}
【拍照,红圈是库伦微扰】
1s 2s 的库伦微扰更小,1s 2p 离原子核更远,受到的屏蔽更小一些,那么【】库伦简并就消除了。如果考虑了电子的相互作用,【】,

在 1s 电子存在的时候,2s 电子 [][]
【两个说法角度不同】

库伦排斥可以从一个简并态分离成两个简并态。

\textbf{3. 交换简并的消除:}
\begin{align}
    \psi_1^{(0)} = 1s(1)\,2s(2) , \quad 
    \psi_2^{(0)} = 2s(1)\,1s(2),
\end{align}
% 本章用的 psi 应该是 phi 和 Phi
\begin{align}
    \Psi = []
\end{align}
不能区分【】?背后对应着,线性变分【】。波函数必须能够反映量子力学体系等同粒子的不可区分性。

我们知道一级微扰的能量
\begin{align}
    E_1^{(1)} = J_{1s\,2s} - K_{1s\,2s}, E_2 [][]
\end{align}
得到
\begin{align}
    E_1^{(1)} - E_2^{(2)} = - 2K_{1s\,2s}
\end{align}
称作交换简并消除。

% 2022-11-28 14:37:43  Wenbin Fan @FDU
交换简并消除意味着什么?为什么 $E^{(1)}_1$ 比 $E^{(1)}_2$ 小?当 $r_1 = r_2$ 时,
\begin{align}
    &\Psi_1^{(1)} (r_1, r_2 = r_1) = 0, \\
    &\Psi_2^{(1)} (r_1, r_2 \neq r_1) \neq 0, \Rightarrow [][]
\end{align}
通过构造,使得两个粒子不可能出现在同一个位置。使得体系能量更稳定。这个假设不能保证【】,产生了较大的库伦排斥能。相减的波函数,比相加的更稳定,因为相减的波函数限制了【】
% 2022-11-28 14:40:59  Wenbin Fan @FDU
导致更稳定,后面也会学到。

为什么交换这么重要?下节课讲自旋和多电子体系,有不可区分性。本课接下来的几周没有新的量子力学内容。等同粒子有没有不可区分性?玻色子、费米子,从而引出了自旋的概念。

变分自然地能保证等同粒子的不可区分性。

% 2022-11-28 14:42:53  Wenbin Fan @FDU
我们讲变分的时候。变分是啥意思?给定尝试波函数,永远比基态能量高。从微扰角度来说,从 -68 eV 围绕上去,一直得到各个态。但是,高级别微扰得到的 -59.2 eV 比 -68 eV 高,有没有破坏变分原理呢?这个 -68 eV 是 $H_0$ 的本正函数,对应于 < | H0 | > = E1 ,是我们对哈密顿进行处理的,哈密顿量没有近似。微扰中是近似哈密顿,近似哈密顿的基态能量,不能保证比【】能量要高。【】不考虑排斥作用,这个哈密顿的好处是什么呢,不含库伦排斥的哈密顿量求出的能量更低。变分原理在微扰【】不保证,期望值不比体系的基态能量高。【返回 非简并微扰理论】

假设我们把【】代回到真实的【】?
\begin{align}
    [][]
\end{align}
能否保证?没做变分。以经验来说,偶数阶的微扰能量更低,奇数项能量更高,所以量化计算中常用的微扰理论方法是二阶微扰 MP2、二阶三阶混合的 MP2.5。

\homework{\textbf{10.2}  氢原子受 $z$ 轴方向均匀外加电场微扰时,其微扰 $\hat H' \equiv e \vec A = e A \, r \cos\theta$,其中 $A$ 是电场强度。考察 $\hat H'$ 对 $n=2$ 能级的影响。这个能级是四重简并。因为 $\hat H'$ 与角动量算符 $\hat L_z$ 是对易的,可利用 2p$_{0}$、2p$_{-1}$ 建立久期行列式,并利用宇称证明部分 $\hat H'$ 矩阵元素为零。求非零积分、能量一级校正、正确的零级波函数。}
这个题目比氦原子稍微简单,这里只有一个电子,从 1s 激发到 2s、2p 上。

到这里,微扰算是讲完了。我们涉及到的粒子只有两个,【】定义,时间有限,【】

%2022-11-28 14:56:55  Wenbin Fan @FDU
原来我想讲的是密度泛函理论。一旦讲到 DFT 就涉及到多电子体系,【】

大家听过密度泛函理论吗?字面理解是啥意思?(同学:存在定理)八九不离十都能讲点东西。泛函是什么意思?函数的函数,输入一个函数得到数值,那么函数是数值到数值的映射。DFT 里的泛函是波函数作为输入,我们要找到一个波函数让能量最低。

密度泛函,是指用密度映射能量,基态能量可以由基态密度唯一确定。这话有什么好处?我们求薛定谔方,不考虑自旋、时间,波函数是 $N$ 个电子 $3N$ 维度的变量。粒子是不可区分的,在任一【】全空间有 $N$ 个电子,密度的全空间积分为 $N$。密度的求解远比薛定谔方程求解简单。如果能找到映射关系【】,那么就有更简单的方式求解薛定谔方程。这就是我希望把 DFT 放在变分微扰后面讲的原因。变分是寻找尝试波函数,微扰是利用未微扰波函数作为跳板。DFT 是一种新的形式【】。

% 2022-11-28 15:03:43  Wenbin Fan @FDU
通过变分看看有没有这种可能性。限制性变分,
\begin{align}
    \langle \psi | \hat H | \psi \rangle |_{\text{min}} = E_0
\end{align}
【】假设知道全空间【?】

% 2022-11-28 15:04:51  Wenbin Fan @FDU
密度相同的波函数,构成子空间 $\{\psi_{\rho_1}\}, \{\psi_{\rho_2}\}, \cdots$,它们构成了 $\{\{\psi_{\rho_1}\}\} = \{\psi\}$。
% 2022-11-28 15:06:26  Wenbin Fan @FDU
找所有的最小。

% 2022-11-28 15:07:14  Wenbin Fan @FDU
确实可以找出来 $E_0$,

把波函数再展开,
\begin{align}
    E[\rho] = 
\langle    \psi_\rho | \hat T + V_{ee} + V_{ext} | \psi_\rho \rangle | _{\text{min}} = [][] 
= F[\rho]
\end{align}
因为 min 走遍了所有密度相同的波函数。动能算符和【】构成了普适泛函。通过限制性搜索,证明密度到能量的映射关系是存在的,一旦映射关系找到了,我们要去搜索所有的密度使得能量最低。如果能够早一个泛函,使得该泛函输出给定密度的最小值能量,那么对密度进行变分可得到体系的静态能量。这个有点超纲了,【】
如何从密度给出【】

% 2022-11-28 15:10:51  Wenbin Fan @FDU
变分【】,微扰【】,密度泛函是为了找到更好的映射关系,进而操作密度而不是波函数得到体系精确的能量。DFT 不对波函数进行操作,但密度不是波函数,密度不包含体系的所有信息,密度不包含微妙的量子耦合等,相当于对体系的全部信息做了取舍。天下没有免费的午餐 no-free-lunch theorem。

上课上到 Dec 19 Mon。还有三次课。我跟大家保证,如果平时作业认真做,最后一定会过的。考试难度只有平时的 70 80\%。本课的出发点不是像高考一样选拔。希望同学们理解化学中背后的物理,当然这离不开数学基础。
