\courseTime{12th, week 15, Dec. 12}
\section{复习}
自旋是粒子的内禀属性,没有宏观对应的物理量。在相对论框架中,自旋是自然产生的。

我们不知道自旋的具体表示。设定自旋角动量的算符为
\begin{align}
    \hat S = \hat S_x \vec i + \hat S_y \vec j + \hat S_z \vec k, 
    \quad 
    \hat S^2 = \hat S_x^2 + \hat S_y ^2 + \hat S_z^2,
\end{align}
通过其满足的角动量对易形式
\begin{align}
    \hat S \times \hat S = \ii \hbar \hat S,
\end{align}
去推导本征值
\begin{align}
    &\hat S^2 = s(s+1)\hbar^2, \quad s = 0, \frac12, 1, \frac32, \cdots, \\
    &\hat S_z = m_s \hbar, \quad m_s = -s, -s+1, \cdots, s-1, s. 
\end{align}
实验证明,所有电子只有单一的自旋大小 $s=1/2$,因此 $m_s = -\frac12, \frac12$。
通常定义
\begin{align}
    |\alpha\rangle = \left|m_s = \frac12 \right\rangle = \begin{pmatrix}
        1\\0
    \end{pmatrix},
    \quad 
    |\beta\rangle = \left|m_s = -\frac12 \right\rangle = \begin{pmatrix}
        0 \\ 1
    \end{pmatrix},
\end{align}
可得自旋角动量的正交归一性
\begin{align}
    \langle\alpha | \beta \rangle = 0, \quad
    \langle \alpha | \alpha \rangle = \langle \beta | \beta \rangle = 1. 
\end{align}

考虑自旋后,氢原子的波函数将含有自旋部分,
\begin{align}
    \psi(x,y,z) \, \chi(m_s), \quad \chi(m_s) = \alpha(m_s), \beta(m_s),
\end{align}
% 记号 g 是哪里来的?一般用 \chi 表示,\chi 即卡方验证中的「卡」
能量不变,因为
\begin{align}
    \hat H[\psi(x,y,z) \chi(m_s)] = \chi(m_s) \, \hat H \psi(x,y,z) = \chi(m_s) \, E \psi(x,y,z). 
\end{align}
\chat{%
目前为止,哈密顿算符中的坐标与自旋是无关的。只有在相对论框架中,或者实际中当相对论效应变得不可忽略时,轨道与自旋会产生耦合,称为 spin-orbital coupling。此时,角动量和自旋角动量均不是好量子数,另定义 $\hat J = \hat L + \hat S$ 为总角动量,$\hat J$ 是好量子数。
}
自旋在非相对论框架中,产生的唯一影响是\textbf{状态数倍增}。

\section{全同粒子}
某微观体系有 $N$ 个电子,先假定它们可以被区分和编号,其中第 $i$ 个电子的坐标和自旋为 $(x_i, y_i, z_i, m_{s,i})$,简记为 $(\vec r_i, m_{s,i})$。用 $q_i$ 表示全部信息,那么波函数可记作 $\psi(q_1, q_2, \cdots, q_N)$,概率密度为 $|\psi(q_1, q_2, \cdots, q_N)|^2$。

由于电子的不可区分性质,交换任意两个电子,概率密度不变,
\begin{align}
    |\psi(q_1, q_2, \cdots, 
    \colorbox{fudanBlue!10!white}{$q_i$}
    , \cdots, 
    \colorbox{fudanRed!10!white}{$q_j$}
    , \cdots, q_N)|^2 
    = 
    |\psi(q_1, q_2, \cdots, 
    \colorbox{fudanRed!10!white}{$q_j$}, \cdots, 
    \colorbox{fudanBlue!10!white}{$q_i$}
    , \cdots, q_N)|^2.
\end{align}
两边同时开根号,只有正负两种结果。若为正,则称作\textbf{对称} symmetry,否则为\textbf{反对称} anti-symmetry。
% 这种约束不是来自于波函数自身。

定义置换算符 $\hat{\mathcal{P}}_{ij}$,其表示交换第 $i$ 个和第 $j$ 个电子,有
\begin{align}
    \hat{\mathcal{P}}_{ij} \, \psi(q_1, \cdots, q_i, \cdots, q_j, \cdots, q_N) = C \, \psi(q_1, \cdots, q_j, \cdots, q_i, \cdots, q_N). 
\end{align}
其中 $C$ 是置换算符的本征值。这个本征值非常容易求解,将同一个置换算符\textbf{两次}作用在波函数上,得到波函数本身,有
\begin{align}
    \hat{\mathcal{P}}_{ij}\hat{\mathcal{P}}_{ij} \psi = C^2 \psi = \psi,
\end{align}
因此 $C = \pm1$。
\suppInfo{置换算符的对易关系}{
我们对电子的标号不影响波函数自身。如果 $\psi$ 是 $\hat H$ 的解,那么 $\hat{\mathcal{P}}_{ij} \psi$ 同样也是 $\hat H$ 的解,由此写出来薛定谔方程
\begin{align}
    \ii\hbar\,\pdv{t} \psi = \hat H \psi,
    \quad
    \ii\hbar\,\pdv{t} \hat{\mathcal{P}}_{ij}\psi = \hat H \,\hat{\mathcal{P}}_{ij}\psi, 
\end{align}
将左式左乘 $\hat{\mathcal{P}}_{ij}$,减去右式,可得对易关系
\begin{align}
    [\hat{\mathcal{P}}_{ij}, \hat H] = 0.
\end{align}
% ref: 余寿绵编著. 高等量子力学[M]. 济南:山东科学技术出版社, 1985.11. 186. 
类似也可证明
\begin{align}
    [\hat{\mathcal{P}}_{ij}^n, \hat{\mathcal{P}}_{ij}^m] = 0, \quad n,m \in \mathbb{Z}. 
\end{align}
}

\subsection{第五个量子力学公设}
由此可以推断,自然界中的一切粒子必然具有对称或反对称的其中一种属性,这种属性是由粒子内禀性质决定的。因此,由对称性划分得到两类粒子。

全同粒子相关的实验和理论表明,自然界中的这两种粒子,存在着与自旋相关的不同统计性质。自旋是\textbf{半整数}的粒子服从 Fermi--Dirac 分布,称为\boldtext{费米子},如电子、质子、中子、$\mathrm{\mu}$-介子等。自旋是整数的粒子服从 Bose--Einstein 分布,称为\boldtext{玻色子},如光子、$\mathrm{\pi}$-介子等。

这两种粒子的行为完全不同,无数个玻色子可以处在同一个状态,但是四个量子数表征的费米子只能有一个。早在量子力学建立前的 1923 年,物理学家泡利 Pauli 就发现了这个特征,此即「\boldtext{泡利原理}」,又叫做「不相容原理」。%
% ref: (苏)索科洛夫(А.А.Соколов)著;王祖望译. 量子力学原理及其应用[M]. 上海:上海科学技术出版社, 1983.05.   459. 

\begin{theorem}[公设5]
    电子体系的波函数,若交换任意两个电子,必须是反对称的,记作
    \begin{align}
        \psi(q_1, q_2, \cdots, q_N) = -\psi(q_2, q_1, \cdots, q_N).
    \end{align}
\end{theorem}

\chat{%
按照此假设,如果两个电子有着相同的坐标 $q_1 = q_2$,此时波函数满足
\begin{align}
    \psi(q_1, q_2) = - \psi(q_1, q_2)
\end{align}
波函数必然为零。但是,只要两电子的自旋不同 $m_{s1}, m_{s2}$ 则波函数就不为零了,那么两电子可以处在完全相同的坐标上。因此,泡利原理又被形象地称作「\textbf{费米排斥}」。
}

\subsection{氦原子的波函数对称性}
氦原子基态的电子组态是 1s$^2$,两个电子都在 1s 轨道上,如果不考虑自旋,这种交换对称的波函数 $\psi_0 = \mathrm{1s(1)\, 1s(2)}$ 是不存在的。如果考虑自旋,并且这两个电子有不同的自旋,此时波函数记作
\begin{align}
    \psi_0 = \mathrm{1s(1)\,1s(2)}\, \chi(m_{s1},m_{s2}), \quad
    \chi(m_{s1}, m_{s2}) = -\chi(m_{s2}, m_{s1}),
\end{align}
那么这种状态是可以存在的。经过了一些基础课程的学习,我们知道自旋态应该写成 
$|\!\uparrow\downarrow\rangle$ 的形式,这背后对应的是泡利原理。

\begin{table}[tp]
    \centering
    \caption{氦原子基态四种可能的电子排布及交换后的排布}
    \label{tab:he_permutation_population}
    \begin{tabular}{
        >{\centering\arraybackslash }m{2cm} 
        >{\centering\arraybackslash }m{1.8cm} 
        >{\centering\arraybackslash }m{1.8cm} 
        >{\centering\arraybackslash }m{1.8cm} 
        >{\centering\arraybackslash }m{1.8cm} 
        }
    \hline
    \\[-10pt]
    % \rule{0pt}{100pt}
    \multirow{2}{*}{\shortstack{交换前的\\电子排布}}
    &
    \begin{tikzpicture}[scale=0.7]
        \newcommand\w{0.8}
        \renewcommand\s{0.3}
        \newcommand\labelshift{0.4}
        \newcommand\arrowlenhalf{0.5}
        \newcommand\arrowdelta{0.15}
    
        \draw (0,0) -- (\w,0);
        
        % draw arrow
        \draw[->, thick, -Latex] (\w/2+\arrowdelta, -\arrowlenhalf) -- (\w/2+\arrowdelta, \arrowlenhalf);
        \draw[->, thick, -Latex] (\w/2-\arrowdelta, -\arrowlenhalf) -- (\w/2-\arrowdelta, \arrowlenhalf);
    \end{tikzpicture}
    &
    \begin{tikzpicture}[scale=0.7]
        \newcommand\w{0.8}
        \renewcommand\s{0.3}
        \newcommand\labelshift{0.4}
        \newcommand\arrowlenhalf{0.5}
        \newcommand\arrowdelta{0.15}
    
        \draw (0,0) -- (\w,0);
        
        % draw arrow
        \draw[->, thick, -Latex] (\w/2+\arrowdelta, \arrowlenhalf) -- (\w/2+\arrowdelta, -\arrowlenhalf);
        \draw[->, thick, -Latex] (\w/2-\arrowdelta, \arrowlenhalf) -- (\w/2-\arrowdelta, -\arrowlenhalf);
    \end{tikzpicture}
    &
    \begin{tikzpicture}[scale=0.7]
        \newcommand\w{0.8}
        \renewcommand\s{0.3}
        \newcommand\labelshift{0.4}
        \newcommand\arrowlenhalf{0.5}
        \newcommand\arrowdelta{0.15}
    
        \draw (0,0) -- (\w,0);
        
        % draw arrow
        \draw[->, thick, -Latex] (\w/2+\arrowdelta, \arrowlenhalf) -- (\w/2+\arrowdelta, -\arrowlenhalf);
        \draw[->, thick, -Latex] (\w/2-\arrowdelta, -\arrowlenhalf) -- (\w/2-\arrowdelta, \arrowlenhalf);
    \end{tikzpicture}
    &
    \begin{tikzpicture}[scale=0.7]
        \newcommand\w{0.8}
        \renewcommand\s{0.3}
        \newcommand\labelshift{0.4}
        \newcommand\arrowlenhalf{0.5}
        \newcommand\arrowdelta{0.15}
    
        \draw (0,0) -- (\w,0);
        
        % draw arrow
        \draw[->, thick, -Latex] (\w/2+\arrowdelta, -\arrowlenhalf) -- (\w/2+\arrowdelta, +\arrowlenhalf);
        \draw[->, thick, -Latex] (\w/2-\arrowdelta, +\arrowlenhalf) -- (\w/2-\arrowdelta, -\arrowlenhalf);
    \end{tikzpicture}
    \\
     &
    $\alpha(1)\,\alpha(2)$ &
    $\beta(1)\,\beta(2)$ &
    $\alpha(1)\,\beta(2)$ &
    $\beta(1)\,\alpha(2)$ \\[5pt]
    % \multicolumn{5}{c}{\emph{交换两个电子}} \\
    \multirow{2}{*}{\shortstack{交换后的\\电子排布}}
    &
    \begin{tikzpicture}[scale=0.7]
        \newcommand\w{0.8}
        \renewcommand\s{0.3}
        \newcommand\labelshift{0.4}
        \newcommand\arrowlenhalf{0.5}
        \newcommand\arrowdelta{0.15}
    
        \draw (0,0) -- (\w,0);
        
        % draw arrow
        \draw[->, thick, -Latex] (\w/2+\arrowdelta, -\arrowlenhalf) -- (\w/2+\arrowdelta, \arrowlenhalf);
        \draw[->, thick, -Latex] (\w/2-\arrowdelta, -\arrowlenhalf) -- (\w/2-\arrowdelta, \arrowlenhalf);
    \end{tikzpicture}
    &
    \begin{tikzpicture}[scale=0.7]
        \newcommand\w{0.8}
        \renewcommand\s{0.3}
        \newcommand\labelshift{0.4}
        \newcommand\arrowlenhalf{0.5}
        \newcommand\arrowdelta{0.15}
    
        \draw (0,0) -- (\w,0);
        
        % draw arrow
        \draw[->, thick, -Latex] (\w/2+\arrowdelta, \arrowlenhalf) -- (\w/2+\arrowdelta, -\arrowlenhalf);
        \draw[->, thick, -Latex] (\w/2-\arrowdelta, \arrowlenhalf) -- (\w/2-\arrowdelta, -\arrowlenhalf);
    \end{tikzpicture}
    &
    \begin{tikzpicture}[scale=0.7]
        \newcommand\w{0.8}
        \renewcommand\s{0.3}
        \newcommand\labelshift{0.4}
        \newcommand\arrowlenhalf{0.5}
        \newcommand\arrowdelta{0.15}
    
        \draw (0,0) -- (\w,0);
        
        % draw arrow
        \draw[->, thick, -Latex] (\w/2+\arrowdelta, -\arrowlenhalf) -- (\w/2+\arrowdelta, +\arrowlenhalf);
        \draw[->, thick, -Latex] (\w/2-\arrowdelta, +\arrowlenhalf) -- (\w/2-\arrowdelta, -\arrowlenhalf);
    \end{tikzpicture}
    &
    \begin{tikzpicture}[scale=0.7]
        \newcommand\w{0.8}
        \renewcommand\s{0.3}
        \newcommand\labelshift{0.4}
        \newcommand\arrowlenhalf{0.5}
        \newcommand\arrowdelta{0.15}
    
        \draw (0,0) -- (\w,0);
        
        % draw arrow
        \draw[->, thick, -Latex] (\w/2+\arrowdelta, +\arrowlenhalf) -- (\w/2+\arrowdelta, -\arrowlenhalf);
        \draw[->, thick, -Latex] (\w/2-\arrowdelta, -\arrowlenhalf) -- (\w/2-\arrowdelta, +\arrowlenhalf);
    \end{tikzpicture}
    \\
    &
    $\alpha(2)\,\alpha(1)$ &
    $\beta(2)\,\beta(1)$ &
    $\beta(2)\,\alpha(1)$ &
    $\alpha(2)\,\beta(1)$ \\[5pt]
    是否可区分 &
    不可 &
    不可 &
    可 &
    可 \\[5pt]
    对称情况 &
    对称 &
    对称 &
    对称 &
    对称 \\
    \hline
    \end{tabular}
\end{table}
\courseTime{2 of 4, 12th, week}
表 \ref{tab:he_permutation_population} 列出了氦原子基态四种可能的自旋排布,及其交换电子坐标后的排布。从表中可以看到,任何两个电子交换之后,整体的波函数依然不变,它们不满足交换反对称。我们穷尽了四种可能的方式,但这些都不满足交换反对称的定义。

回忆微扰求解氦原子激发态的结果,库伦作用将简并能级一分为二,求得一级微扰波函数为两个态的线性组合。因此,仿照该线性组合的方式,构造
\begin{align}
    \chi_1(m_{s1}, m_{s2}) = \frac1{\sqrt 2} [\alpha(1)\beta(2) - \beta(1)\alpha(2)], 
    \label{eq:he_spin_abba}
\end{align}
将置换算符 $\mathcal{\hat P}_{12}$ 作用上去,可得到相反的自旋态,
\begin{align}
    \mathcal{\hat P}_{12} \chi(m_{s1}, m_{s2}) = \frac1{\sqrt{2}} [\beta(1)\alpha(2) - \alpha(1)\beta(2)] = -\chi(m_{s1}, m_{s2}). 
\end{align}
如果两个自旋态相加,
\begin{align}
    \chi_2(m_{s1}, m_{s2}) = \frac1{\sqrt 2} [\alpha(1)\beta(2) + \beta(1)\alpha(2)],
\end{align}
则不满足交换反对称。
% 为什么要做下面这种验证?% 2022-12-12 20:20:54 Wenbin FAN @FDU
容易证明,合并后的两种自旋态是正交归一的,
\begin{align}
    \langle \chi_1 | \chi_2 \rangle = 0, \quad \langle \chi_1 | \chi_1 \rangle = \langle \chi_2 | \chi_2 \rangle = 1. 
\end{align}

由以上推导可知,(\textbf{1}) 氦原子的基态没有简并,(\textbf{2}) 基态的两个电子如果出在相同的空间轨道上,则自旋必然不同。

\section{Slater 行列式}
上述构建反对称波函数的方式可推广至多电子体系,此方式称为 \boldtext{Slater 行列式}。对于氦原子正确的基态自旋波函数 \eqref{eq:he_spin_abba},展开并写成
\begin{align}
    \psi_0=&\frac1{\sqrt2} \bigl[
        \mathrm{1s}(1)\,\alpha(1) \ \mathrm{1s}(2)\, \beta(2) - \mathrm{1s}(1)\,\beta(1)\ \mathrm{1s}(2)\,\alpha(2)
    \bigr] \\
    =& \frac1{\sqrt2} \left|\begin{matrix}
        \mathrm{1s}(1)\,\alpha(1) & \mathrm{1s}(1)\,\beta(1)\\
        \mathrm{1s}(2)\,\alpha(2) & \mathrm{1s}(2)\, \beta(2)
    \end{matrix}\right|
    = \frac1{\sqrt2} \, \mathrm{1s(1)\,1s(2)} 
    \left|
        \begin{matrix}
            \alpha(1) & \beta(1) \\ \alpha(2) & \beta(2)
        \end{matrix}
    \right|.
\end{align}
观察发现,Slater 行列式只有对角元是独立的,只要写出对角元,非对角元也就唯一确定。所以氦原子的行列式可进一步缩写为
\begin{align}
    \psi_0 = |\mathrm{1s}\,\overline{\mathrm{1s}}|,
\end{align}
其中没有横线的轨道为 $\alpha$,加横线的轨道为 $\beta$。
\chat{%
总结一下本次课目前的内容。量子力学公设 5 告诉我们,自旋半整数的电子是费米子,电子波函数必须是反对称的,这一结论与泡利不相容原理可相互印证。

对于两电子体系,同样利用反对称的约束,从多种可能的基态中线性组合得到了唯一一种满足反对称条件的自旋态。

人们从这种构造方法中,总结形成了 Slater 行列式,用行列式的性质描述多电子体系波函数的反对称性质。
}
\subsection{激发态行列式}
% % This subsection was written by ZY Zhu. 
% 从Slater行列式的简写出发构造激发态,最后可以发现展开后的结果是一样的。

仅考虑 1s、2s 上的占据,共有 $C_4^2 = 6$ 个可能的行列式 $\ket{1\bar{1}}$、$\ket{1\bar{2}}$、$\ket{\bar{1}2}$、$\ket{12}$、$\ket{\bar{1}\bar{2}}$、$\ket{2\bar{2}}$,其中 $\ket{1\bar{1}}$是基态,$\ket{2\bar{2}}$ 是较高的激发态(称「双激发」),剩余四个是单激发行列式。

$ \ket{12}$、$\ket{\bar{1}\bar{2}} $显然可以分离出空间部分和自旋部分
\begin{align}
    &D_1= \ket{12} = \frac{1}{\sqrt{2}} \mqty| \mathrm{1s}(1)\,\alpha(1) & \mathrm{2s}(1)\,\alpha(1) \\ \mathrm{1s}(2)\,\alpha(2) & \mathrm{2s}(2)\,\alpha(2) | 
    = \frac{1}{\sqrt{2}} \mqty| \mathrm{1s}(1) & \mathrm{2s}(1) \\ \mathrm{1s}(2) & \mathrm{2s}(2) |\alpha(1)\,\alpha(2), \\
    &D_2 = \ket{\bar{1}\bar{2}}  = 
    \frac1{\sqrt2}\mqty| \mathrm{1s}(1)\,\beta(1) & \mathrm{2s}(1)\,\beta(1) \\ \mathrm{1s}(2)\,\beta(2) & \mathrm{2s}(2)\,\beta(2) |
    = \frac{1}{\sqrt{2}} \mqty| \mathrm{1s}(1) & \mathrm{2s}(1) \\ \mathrm{1s}(2) & \mathrm{2s}(2) |\beta(1)\,\beta(2), 
\end{align}
而另外两个单激发行列式不能分离。它们不是自旋纯态,但是它们可以线性组合出两个自旋纯态
\begin{align}
    D_3 
    &= \frac{1}{\sqrt{2}} [\ket{1\bar{2}} + \ket{\bar{1}2}] \\
    &= \frac{1}{\sqrt{2}} [\mathrm{1s}(1)\,\mathrm{2s}(2) - \mathrm{2s}(1)\,\mathrm{1s}(2)] \frac{1}{\sqrt{2}} [\alpha(1)\,\beta(2) + \beta(2)\,\alpha(1)], \\
    D_4 
    &=\frac{1}{\sqrt{2}} [\ket{1\bar{2}} - \ket{\bar{1}2}] \\
    &= \frac{1}{\sqrt{2}} [\mathrm{1s}(1)\,\mathrm{2s}(2) + \mathrm{2s}(1)\,\mathrm{1s}(2)] \frac{1}{\sqrt{2}} [\alpha(1)\,\beta(2) - \beta(2)\,\alpha(1)].  
\end{align}
我们发现,如果空间部分是反对称的,那么自旋部分就是对称的($D_1, D_2, D_3$),如果空间部分是对称的,那么自旋部分就是反对称的($D_4$)。

\subsection{多电子 Slater 行列式}
一般的多电子行列式
\begin{equation}
    \dfrac{1}{\sqrt{N!}} 
    \mqty| f_1(1) & f_2(1) & \cdots & f_n(1) \\
            f_1(2) & f_2(2) & \cdots & f_n(2) \\
            \vdots & \vdots & \ddots & \vdots \\
            f_1(n) & f_2(n) & \cdots & f_n(n) |
\end{equation}
同样地,我们只关心它的简写
\begin{equation}\label{key}
	\ket{ f_1(1) f_2(2) \cdots f_n(n) }.
\end{equation}
\suppInfo{行列式的性质}{
\begin{enumerate}
\item 如果行列式中,某一行或某一列的元素全部为 0,则这个行列式的值为 0
\item 交换任意两行或两列,行列式的值乘以 $-1$
\item 若行列式的任意两行或两列相同,则行列式的值为 0
\end{enumerate}

行列式的性质恰好可描述反对称的波函数。性质 2 意味着占据轨道的顺序不影响能量,性质 3 告诉我们泡利不相容原理,即自旋相同的波函数为 0,
\begin{align}
\left|\begin{matrix}
\mathrm{1s(1)}\,\alpha(1) & \mathrm{1s(1)}\,\alpha(1) \\
\mathrm{1s(2)}\,\alpha(2) & \mathrm{1s(2)}\,\alpha(2)
\end{matrix}
\right| = 0. 
\end{align}
}

\subsection{三电子:锂原子的基态}
锂原子的哈密顿量为
\begin{align}
    \hat H = \sum_{i=1}^3 \hat T_i + \sum_{i=1}^3 \frac{3e^2}{|\vec r_i - \vec R|} - \sum_{i<j}^3 \frac{e^2}{r_{ij}}.
\end{align}
\chat{%
如果 $\psi_0 = \mathrm{1s(1)\,1s(2)\,1s(3)}$ 当作试探波函数,变分计算得到 $\langle \psi_0 | \hat H | \psi_0 \rangle = \SI{-214.3}{\electronvolt} < E_0$。这一结果看似违背了变分原理,实际上这是有漏洞的,因为试探波函数必须是与待求体系有相同边界条件的品优波函数,这一试探波函数违反了量子力学公设 5。

在两电子体系中,尽管电子不是反对称的,但我们加入自旋项后波函数便满足了反对称的要求。那么在三电子体系中,能否通过加入自旋,使得对称波函数称为反对称波函数呢?

为了满足全同粒子的反对称要求,前两个电子可以是 $\alpha(1)\,\beta(2)$,那么第三个电子只能是 $\alpha(3)$ 或 $\beta(3)$,所以写出这两种情况的 Slater 行列式
\begin{align}
\left|
\begin{matrix}
\alpha(1) & \beta(1) & \alpha(1) \\
\alpha(2) & \beta(2) & \alpha(2) \\
\alpha(3) & \beta(3) & \alpha(3)
\end{matrix}
\right|
, \quad
\left| 
\begin{matrix}
\alpha(1) & \beta(1) & \beta(1) \\
\alpha(2) & \beta(2) & \beta(2) \\
\alpha(3) & \beta(3) & \beta(3)
\end{matrix}
\right|
, \quad 
\end{align}
这两个行列式总有两列是相等的,因此行列式为 0,进而说明这三个电子不可能全部都在 1s 轨道。
}

锂原子的三个电子,两个在 1s 轨道、一个在 2s 轨道,Slater 行列式记作
\begin{align}
	\Psi_0 
    &= \frac{1}{\sqrt{3!}} 
	\mqty| 1s(1)\alpha(1) & 1s(1)\beta(1) & 2s(1)\alpha(1) \\
	       1s(2)\alpha(2) & 1s(2)\beta(2) & 2s(2)\alpha(2) \\
	       1s(3)\alpha(3) & 1s(3)\beta(3) & 2s(3)\alpha(3) |
    \\ 
	&= \frac1{\sqrt6} \left|
        \begin{matrix}
            1s(1) & \overline{1s}(1) & 2s(1) \\
            1s(2) & \overline{1s}(2) & 2s(2) \\
            1s(3) & \overline{1s}(3) & 2s(3)
        \end{matrix}
    \right|
    = \ket{1s \, \overline{1s} \, 2s}. 
\end{align}

\chat{%
泡利不相容原理,又可以表述成,没有两个电子能占据同一个状态。

由能量最低排布原则、波函数反对称、泡利不相容原理,我们能推导出元素周期表中主族元素的大部分规律。部分副族金属由于强关联、相对论效应等会使得 s、d 轨道有变化。
}

锂原子的基态能量表示为
\begin{align}
    \psi_1 = |\chi_1\,\chi_2\,\chi_3| = \frac1{\sqrt6} 
    \left|\begin{matrix}
        \colorbox{fudanBlue!10!white}{$\chi_1(1)$} & \chi_2(1) & \chi_3(1) \\
        \chi_1(2) & \colorbox{fudanBlue!10!white}{$\chi_2(2)$} & \chi_3(2) \\
        \chi_1(3) & \chi_2(3) & \colorbox{fudanBlue!10!white}{$\chi_3(3)$}
    \end{matrix}
    \right|,
\end{align}
\chat{%
从矩阵的二维视角来看,第一个电子可以出现在 3 个轨道之一上、第一个轨道上也可能出现 3 个电子之一。从对角元来看,本质上就是三个轨道。

我们构造好了满足反对称性质的尝试波函数,接下来求能量。
}
\courseTime{3 of 4}
\subsection{锂原子的能量}
将锂原子的哈密顿量表示为\textbf{单电子轨道}的加和
\begin{align}
    \hat H = \sum_{i=1}^3 \hat h_i - \sum_{i<j}^3 \frac{e^2}{r_{ij}},
    \quad
    \hat h_i = \hat T_i + \frac{3e^2}{|\vec r_i - \vec R|^2},
\end{align}
待求能量 $E[\psi] = \langle \psi | \hat H | \psi \rangle$ 同样也与各轨道能量有关。

展开自旋波函数,
\begin{align}
    \psi_1&= 
    \frac1{\sqrt6} \left(
        \chi_1(1) \left|\begin{matrix}
            \chi_2(2) & \chi_3(2) \\
            \chi_2(3) & \chi_3(3)
        \end{matrix}
        \right|
        -
        \chi_2(1) \left|\begin{matrix}
            \chi_1(2) & \chi_3(2) \\
            \chi_1(3) & \chi_2(3)
        \end{matrix}
        \right|
        +
        \chi_3(1) \left|\begin{matrix}
            \chi_1(2) & \chi_2(2) \\
            \chi_1(3) & \chi_2(3)
        \end{matrix}
        \right|
    \right)
    \notag
    \\
    & =\frac{1}{\sqrt{6}}\big[\chi_1(1) \chi_2(2) \chi_3(3)-\chi_1(1) \chi_3(2) \chi_2(3)\notag \\
    & \phantom{=\sqrt6\big[}-\chi_2(1) \chi_1(2) \chi_3(3)+\chi_2(1) \chi_3(2) \chi_1(3) \notag \\
    &\phantom{=\sqrt6\big[}+\chi_3(1) \chi_1(2) \chi_2(3)-\chi_3(1) \chi_2(2) \chi_1(3) \big],
\end{align}
\chat{%
量化中常用的积分形式为
\begin{align}
    \int \frac{f^*(1)g(1)\,m^*(2)n^*(2)}{r_{12}}\dd\tau_1\,\dd\tau_2 = (fg|mn) = \langle fm | gn \rangle,
\end{align}
其中 $f,g,m,n$ 均表示轨道。圆括号是量化中的常见记法,竖线左侧表示第一个电子、右侧表示第二个电子,且第一个字母为共轭、第二个字母为正常。尖括号是 Dirac 记号,竖线左侧表示共轭。
}
锂原子的能量可表示为
\begin{align}
\langle \psi | \hat H | \psi \rangle =&{} 
\langle \chi_1(1) | \hat h(1) | \chi_1(1) \rangle + 
\langle \chi_2(1) | \hat h(1) | \chi_2(1) \rangle + 
\langle \chi_3(1) | \hat h(1) | \chi_3(1) \rangle  \notag \\
&{}+(\chi_1\chi_1|\chi_2\chi_2) + (\chi_1\chi_1|\chi_3\chi_3) + (\chi_2\chi_2|\chi_3\chi_3) \notag \\
&{}-(\chi_1\chi_2|\chi_2\chi_1) + (\chi_1\chi_3|\chi_3\chi_1) + (\chi_2\chi_3|\chi_3\chi_2)
\end{align}
\chat{%
这里有两种积分,分别是库伦积分和交换积分,
\begin{align}
&(\chi_1\chi_1|\chi_2\chi_2) = \int \frac{\chi_1^*(1)\chi_1(1)\,\chi_2^*(2)\chi_2(2)}{r_{12}} \,\dd\tau_1 \,\dd\tau_2, \\
&(\chi_1\chi_2|\chi_2\chi_1) = \int \frac{\chi_1^*(1)\chi_2(1)\,\chi_2^*(2)\chi_1(2)}{r_{12}} \,\dd\tau_1 \,\dd\tau_2,
\end{align}
库伦积分有明确的物理意义,它表示两个轨道(或者说是电子云)之间的相互作用,但是交换积分并没有可供参考的物理量对应。交换积分完全是由于费米子的反对称性产生的,是纯粹的量子效应。

再定义一种积分记法,
\begin{align}
    (fg|mn)-(fn|mg) = (fg||mn)
\end{align}
}
简记为
\begin{align}
    \langle \psi | \hat H | \psi \rangle ={}& 
    \langle \chi_1(1) | \hat h(1) | \chi_1(1) \rangle + 
    \langle \chi_2(1) | \hat h(1) | \chi_2(1) \rangle + 
    \langle \chi_3(1) | \hat h(1) | \chi_3(1) \rangle  \notag \\
    &{}+(\chi_1\chi_1||\chi_2\chi_2) + (\chi_1\chi_1||\chi_3\chi_3) + (\chi_2\chi_2||\chi_3\chi_3),
\end{align}
\chat{%
再对能量的形式做一些简化。引入
\begin{align}
    h_{ig} = \langle \chi_i | \hat h_i | \chi_j \rangle,
\end{align}
则能量的前三项简记为 $\sum_i h_{ii}$。后三项看起来与矩阵表示有关,令 $(\chi_i\chi_i || \chi_j\chi_j) = (ii||jj)$,则后三项为
\begin{align}
% \frac12 
% \left|
\begin{matrix}
    0 & (11||22) & (11||33) \\
    (22||11) & 0 & (22||33) \\
    (33||11) & (33||22) & 0
\end{matrix}
% \right|,
\end{align}
为了形式上更好看,我们考虑补全对角元,正巧
\begin{align}
    (\chi_i\chi_i||\chi_i\chi_i) = (\chi_i\chi_i|\chi_i\chi_i) - (\chi_i\chi_i|\chi_i\chi_i) = 0,
\end{align}
那么后三项最终记为
\begin{align}
    % \frac12
    % \left|
    \begin{matrix}
        (11||11) & (11||22) & (11||33) \\
        (22||11) & (22||22) & (22||33) \\
        (33||11) & (33||22) & (33||33)
    \end{matrix}
    % \right|,
\end{align}
}
进一步简写为
\begin{align}
    \langle \psi | \hat H | \psi \rangle = \sum_i h_{ii} 
    + \frac12 \sum_i^3 \sum_j^3 (ii||jj),
\end{align}
% \homework{\textbf{12.1}}

\subsection{多电子的 Slater 行列式}
更一般地,对于 $N$ 电子的 Slater 行列式 $\psi = |\chi_1\chi_2\cdots \chi_N$,能量为
\begin{align}
    E[\psi] = E[{\chi_i}] = 
    \sum_i^N h_{ii}
    + \frac12 \sum_i^N \sum_j^N (ii||jj), 
    \label{eq:NSlater_det}
\end{align}
后一项可拆成库伦积分和交换积分,
\begin{align}
    (ii||jj) = (ii|jj) - (ij|ji) = J_{ij} - K_{ij}. 
\end{align}
这是比较符合物理直觉的写法,即体系能量来自于每个轨道的单电子积分和轨道两两之间的库仑和交换积分。

目前我们已经将锂原子能量写成了由单电子轨道构成的尝试波函数的期望值。由变分原理,对能量做变分 $\pdv{E[{\chi_i}]}{\chi_i}$,可以找到使得能量最低的一组电子轨道 $\{\chi_i\}$,这组轨道 $\{\chi_i\}$ 对应的 Slater 行列式正是最低的能量期望值。
% \homework{\textbf{12.1} ~ }

\chat{%
%% This supp info was written by ZY ZHU. 
进一步地,$ i,j $属于不同自旋时,
\begin{align}
 K_{ij} &= (12|21) = \iint\dd\tau_1\dd\tau_2 \chi_1(1)\alpha(1) \chi_2(1)\beta(1) \dfrac{1}{r_{12}} \chi_2(2)\beta(2) \chi_1(2)\alpha(2) \notag\\
 &= \qty[\iint\dd\tau_1\dd\tau_2 \chi_1(1) \chi_2(1) \dfrac{1}{r_{12}} \chi_2(2) \chi_1(2)]\ev{\alpha(1)|\beta(1)}\ev{\beta(2)|\alpha(2)} \notag\\
 &= 0
\end{align}
所以交换积分只需对相同自旋求和即可。

进一步地,如果偶数个电子的Slater行列式中,每一对$ \alpha,\beta $电子的空间部分都是一样的,可以证明($ i,j $对空间轨道求和)\footnote{参考 Szabo \S 2.3.5---2.3.7.}
\begin{equation}
E = 2\sum_i^{N/2} h_{ii} + \sum_{ij}^{N/2} (2J_{ij} - K_{ij})
\end{equation}
例如Be原子$ \ket{\mathrm{1s} \, \overline{\mathrm{1s}} \, \mathrm{2s} \, \overline{\mathrm{2s}}} $,
\begin{align}
E &= 2h_{\mathrm{1s}\,\mathrm{1s}} + 2h_{\mathrm{2s}\,\mathrm{2s}} + 2J_{\mathrm{1s}\,\mathrm{1s}} + 2J_{\mathrm{2s}\,\mathrm{2s}} + 2J_{\mathrm{1s}\,\mathrm{2s}} + 2J_{\mathrm{2s}\,\mathrm{1s}} - K_{\mathrm{1s}\,\mathrm{1s}} - K_{\mathrm{2s}\,\mathrm{2s}} - K_{\mathrm{1s}\,\mathrm{2s}} - K_{\mathrm{2s}\,\mathrm{1s}} \notag\\
&= 2h_{\mathrm{1s}\,\mathrm{1s}} + 2h_{\mathrm{2s}\,\mathrm{2s}} + J_{\mathrm{1s}\,\mathrm{1s}} + J_{\mathrm{2s}\,\mathrm{2s}} + 4J_{\mathrm{1s}\,\mathrm{2s}}  - 2K_{\mathrm{1s}\,\mathrm{2s}}
\end{align}
}
\section{Hartree--Fock}
对多电子的 Slater 行列式 \eqref{eq:NSlater_det} 做变分,可得到 Hartree--Fock 方程
\begin{align}
    \left[
        \hat h(1) + \sum_j^N \bigl(\hat J_j(1) - \hat K_j(1)\bigr)
    \right] \chi_i(1) = \mathcal{E}_i \chi_i(1),
\end{align}
其中库伦和交换算符的定义是
\begin{align}
    \hat J_j(1) \chi_i(1) &= \int\dd\tau_2 \ \chi_j^*(2) \frac{e^2}{r_{12}} \chi_j(2) \chi_i(1),\\
    \hat K_j(1) \chi_i(1) &= \int\dd\tau_2 \ \chi_j^*(2) \frac{e^2}{r_{12}} \chi_i(2) \chi_j(1), 
\end{align}
定义 Fock 算符
\begin{align}
    \hat f = \hat h + \sum_j (J_j - K_j),
\end{align}
得到形似单电子薛定谔方程的公式
\begin{align}
    \hat f_i \chi_i(1) = \mathcal{E}_i \chi_i(1). 
\end{align}
通过求解上式,可再由 Slater 行列式给出体系的最低能量及对应的波函数。

接下来简化库伦和交换算符。注意到交换算符 $\hat K$ 的最后两个轨道是 $ji$,如果用置换算符 $\hat P_{12}$ 交换两个电子,那么得到交换算符为
\begin{align}
    \hat K_j(1) \chi_i(1) = \int\dd\tau_2 \ \chi_j^* \frac{e^2}{r_{12}} \hat P_{12} \bigl(\chi_j(2)\chi_i(1)\bigr).
\end{align}
此时库伦算符和交换算符可以相减,得到
\begin{align}
    \bigl(\hat J_j(1) - \hat K_j(1)\bigr) \chi_i(1) = 
    \int\dd\tau_2 \ \chi_j^*(2) \frac{e^2}{r_{12}} \, (1-\hat P_{12}) 
    \chi_j(2) \chi_i(1),
\end{align}
并由此得到双电子积分
\begin{align}
    \langle \chi_i | \hat J_j(1) - \hat K_j(1) | \chi_i \rangle =
    (ii|jj) - (ij|ji) = (ii||jj). 
\end{align}

\chat{%
并不是一组随便什么样的轨道都是 Fock 算符的本征函数,我们需要\textbf{迭代}使得 Fock 算符的
表示矩阵对角化。实际上的迭代计算是通过将分子轨道 molecular orbital (MO) 展开为原子轨道 atomic orbital (AO) 后的 Hartree--Fock--Roothaan 方程进行的。\footnote{参见 Szabo \S 3.4}

这种「初猜——迭代——收敛」的计算方式称为自洽场 self-consistent field (SCF) 迭代。
}
假设我们已经做完了 SCF 迭代并且收敛,可得到轨道能量
\begin{align}
    \mathcal{E}_i = \langle \chi_i | \hat f | \chi_i \rangle 
    = h_{ii} + \sum_j (ii||jj). 
\end{align}

\subsection{Hartree--Fock 平均场近似}
将 Fock 算符构成体系的零级哈密顿量,也就是把 $N$ 个电子的 Fock 算符线性加和,
\begin{align}
    \hat H_0 = \sum_i \hat f_i,
\end{align}
此时解出的本征函数是这些轨道的 Slater 行列式的线性组合,能量为
\begin{align}
    \hat H_0 \psi = E_0 \psi = \sum_i^{\text{occ}} \mathcal{E}_i\psi. 
\end{align}

将真实哈密顿量与零级近似哈密顿量之间的差别视作微扰,
\begin{align}
    \hat H' = \hat H - \hat H_0,
\end{align}
一级微扰能量为
\begin{align}
    E_i^{(1)} = \langle \psi_i | \hat H' | \psi_i \rangle,
\end{align}
所以真实能量为
\begin{align}
    E \approx E_0 + E^{(1)} = \sum_i^{\text{occ}} \mathcal{E}_i + \langle \psi | \hat H' | \psi \rangle,
\end{align}
称为 Hartree--Fock 能量。

\extraInfo{助教补充}{% ZY ZHU, SR WANG
如对本节课的内容感到困惑,通读 Szabo §2.1-2.3, §3.1-3.5 也许能解决很多问题。Szabo 教材的习
题解答可以参考 \href{https://github.com/hebrewsnabla/S-O-MQC-HW}{Solutions for Modern Quantum Chemistry, Szabo \& Ostlund}. 

本章节的许多 $N$ 电子 Slater 行列式的结论并不容易证明,但是仅对两三个电子的情况是容易验证的。
}
