\section{章节总结}
\courseTime{Sep. 19, 2022 - Week 3 - }
\chat{%
上周我们回顾了一下量子力学的诞生,这周课我们会讲算符、量子力学的公设。这这节课与物化AI的内容其实差不了太多,但是我们试图重新梳理,不是推导,演绎量子力学的发展历程,希望大家能加深理解。

如果大家做作业,包括对课程有任何的问题跟建议,都可以跟我们联系。

我们先来回顾上节课的内容,然后做一些基本的总结。因为每节课我也经常控制不好,有时想讲东西太多。事实上,有的关键点反而可能还会漏掉一些,所以我现在会来补一下。

上节课我们讲的,总结起来只有一页纸的内容,即量子力学的诞生。量子力学诞生,根本的地方其实就是在于对物质波粒二象性的理解,特别在微观世界中。}

波粒二象性最重要的关系就是 Planck--Einstein 关系式,
\begin{equation}
    E = h\nu = \hbar \omega
\end{equation}
这个公式,最早是由普朗克在拟合黑体辐射公式时,得到的一个新的物理概念。他发现,只有假设电磁波的辐射的话是一份一份进行的,才能拟合黑体辐射的频率(波长)跟辐射能之间的关系。

\chat{%
最重要的是说,光、电磁波,波的能量必须是一份一份的,所以每份能量跟波的频率直接成正比,并且这个相关的系数是一个常数,我们叫它 Planck 常量。后面经常会遇到 $\hbar$,$\hbar$对应的就是角频率角频率 $\omega$。
我们讲,一个粒子或者说一个东西,它是粒子。除了它的能量,那我们知道它的运动的话离不开动量。一个粒子,它的运动有方向、动量,并且动量也要满足动量守恒的规则。所以它必须有一个动量的表达式。动量的这个推导能证明,量子力学的诞生要基于相对论的诞生。没有相对论,我们没有办法直接给出光动量的方程式的。那么它到底是怎么来的呢?
}

在相对论的框架下面,有个质能关系。其中最重要的想法是,质量的变化满足
\begin{eqnarray}
    m = m_0 + \Delta m,
\end{eqnarray}
对于光而言的话,在相对论的框架下,
\begin{eqnarray}
    m = \frac{m_0}{\sqrt{1-\left(\frac vc\right)^2}}. 
\end{eqnarray}
所以当速度接近于光速的时候,$\Delta m \rightarrow \infty$。所以在相对论的框架下,光的初始质量一定是要为零,才能保证运动质量是常数。

对于任何物体或粒子,均有
\begin{equation}
    \left\{
        \begin{aligned}
            &\Delta E = \Delta m c^2,\\
            &p = mv,
        \end{aligned}
    \right.
\end{equation}
对于光子,
\begin{equation}
    \left\{
        \begin{aligned}
            &E = mc^2, \\
            &p = mc = \frac Ec = \frac{h\nu}c = \frac h\lambda, 
        \end{aligned}
    \right.
\end{equation}
可推知
\begin{equation}
    \left\{
        \begin{aligned}
            &E = h \nu = \hbar\omega , \\
            &p = \frac{h}{\lambda} = \hbar k, 
        \end{aligned}
    \right.
    \label{eq:photon_Ep}
\end{equation}
进而有
\begin{equation}
    \left\{
        \begin{aligned}
            &\omega = 2\pi\nu,
            &k = \frac{2\pi}{\lambda}, 
        \end{aligned}
    \right.
\end{equation}
于是 $\omega = ck$ 是线性关系。

\chat{%
到了实物波,我们通常处理的是在非相对论框架下的这个微观粒子——电子。目前不涉及重金属、过渡金属,特别是 F 区、镧系、锕系金属中的电子。我们通常处理氢元素中的电子,它的运动的速度是远低于光速的。
如何理解粒子的实物波?必须提到德布罗意的做法。他认为,实物也有波粒二象性,电子的干涉、衍射证明了波动的一面。波的性质是如何表述的呢?德布罗意认为,将普朗克-爱因斯坦的关系直接应用到实物波中。
}

以电子、质子为代表的微观粒子,静质量不为零 $m_0 \neq 0$,运动速度远低于光速导致 $\Delta m \simeq 0 |_{v \ll c}$,那么能量
\begin{eqnarray}
    E = \frac{p^2}{2m}.
\end{eqnarray}
\chat{%
这个公式是我们从牛顿力学就知道的。如果我们认为波动性跟粒子性是微观粒子的一体两面,那么它必定有波动的性质。那波动的性质所对应的物质波的角频率、物质波的波矢,跟粒子性是什么关系?}
德布罗意把这两种表述强行耦合在一起。根据 \eqref{eq:photon_Ep},粒子的动量为
\begin{eqnarray}
    \hbar\omega = \frac{(\hbar k)^2}{2m},
\end{eqnarray}
因此得到 $\omega = \frac{\hbar k^2}{2m}$。
\chat{%
现在只是在演绎,怎么能够把非相对论框架下的薛定谔方程演绎出来。至于相对论框架的狄拉克方程,我们可以做其他方式的拓展。
}

为了更好地描述粒子实物波的粒子,上节课引入了 Gaussian 波包。我们先推导了自由粒子的波包,它的波动方程,
\begin{eqnarray}
    \psi(x,t) = A \exp[\ii (kx - \omega t)],
\end{eqnarray}
将不同波矢的自由粒子做一个全空间的积分或者组合,我们就可以得到一个波包,
\begin{eqnarray}
    \psi(x,t) = \int_{-\infty}^{\infty} A(k) \exp[\ii (k x - \omega(k) t)] \dd k,
\end{eqnarray}
最大的不同与,振幅 $A$、角频率都与波矢有关。

代入波动方程中,显然自由粒子的波包是满足的,
\begin{lstlisting}
\[Psi][x_, t_] := A Exp[I (k x - c k t)];
D[\[Psi][x, t], {x, 2}]/D[\[Psi][x, t], {t, 2}]
>> 1/c^2
\end{lstlisting}
但物质波的波包并不满足,
\begin{eqnarray}
    \frac{\partial^2 \psi}{\partial x^2} = \frac{4 m^2}{k^2 \hbar^2} \frac{\partial^2 \psi}{\partial t^2}. 
\end{eqnarray}
\begin{lstlisting}
\[Psi][x_, t_] := A[k] Exp[I (k x - (\[HBar] k^2)/(2 m) t)];
D[\[Psi][x, t], {x, 2}]/D[\[Psi][x, t], {t, 2}]
>> (4 m^2)/(k^2 \[HBar]^2)
\end{lstlisting}
左右两个偏导并不是线性关系。

上节课推导了物质波的波包。
注意到,如果物质波对坐标求二阶偏导、对时间求一阶偏导,二者是有线性关系的,即
\begin{eqnarray}
    \frac{\partial^2 \psi}{\partial x^2} = - \ii\frac{2m}\hbar \frac{\partial \psi}{\partial t}. 
\end{eqnarray}
稍作变化,便可发现
\begin{eqnarray}
    -\frac{\hbar^2}{2m} \frac{\partial^2 \psi}{\partial x^2} = \ii\hbar \frac{\partial\psi}{\partial t},
    \label{eq:free_partial_schrodinger}
\end{eqnarray}
这是自由粒子的薛定谔方程。这个没有严格推导,只有演绎。

\chapter{算符与量子力学公设}
\chat{%
对于定义,会直接陈述。对于跟宏观世界不太一样的公设、算符,会展开讲。

微观世界中粒子的行为可以用什么波的状态方程来描述,如何得到体系中的可观测量物理量?我们要将算符作用到系统的波函数上才能得到一些信息。

首先,我们必须定义到底什么是算符。从一些常见的算符开始讲起。
}
\section{常见的算符}
对一个具体的波函数,它包含了该体系所有的信息。

那么对于任意一个算符 $\hat A$,相应的物理可观测量要怎么得到?假设算符 $\hat A$ 是时间的函数,定义有
\begin{eqnarray}
    \langle A(t) \rangle = \intinf \psi^*(x,t) \ \hat{A} \ \psi(x,t) \dd x,
\end{eqnarray}
对于位置算符,有
\begin{eqnarray}
    \langle x(t) \rangle = \intinf \psi^*(x,t) \ x\  \psi(x,t) \dd x,
\end{eqnarray}
\chat{%
位置算符就是位置本身,所以位置算符本身不是时间的函数。因为波函数本身包含时间,所以只对实空间积分就给出了波函数随时间演化的平均位置。也就说,在粒子散射的过程中,就可以通过这个方式给出粒子的平均运动轨迹。还是要强调,后面的推导并不是证明,只是演绎,希望能够增加理解。
}

\subsection{动量算符的推导}
\chat{%
大家知道动量算符是什么样子,它为什么是这个样子状态?跟宏观动量有何关系?为何与质量无关?
}

现在我们先回到牛顿力学中,有
\begin{eqnarray}
    p(t) = mv(t) = m \frac{\partial x}{\partial t},
\end{eqnarray}
\chat{%
无论在微观世界、宏观世界,还是在非相对论框架、相对论框架,这个是普适存在的。只不过说,在微观世界里面,如果体系的速度不是本征值的时候,那么我们要对速度求期望值,也就是速度的期望值也是动量的期望值。

在牛顿力学里,速度的描述是坐标,对时间的偏导是速度。如果我们在微观世界里,因为微观世界不确定这体系是不是坐标或动量的本征函数。那怎么办呢?我们就要对它们两个同时做期望值,
}
两个期望一定满足
\begin{eqnarray}
    \langle p(t) \rangle = m \frac{\partial \langle x(t) \rangle}{\partial t},
\end{eqnarray}
这就是动量的公式在微观世界中的描述。

代入定义,有
\begin{eqnarray}
    \langle p(t) \rangle = m \intinf \frac{\partial}{\partial t} [\psi^*(x,t)\ x \ \psi(x,t)] \dd t,
\end{eqnarray}
利用复合函数求导的定义
\begin{eqnarray}
    \frac{\partial}{\partial x} [f(x) g(x)] = \frac{\partial f(x)}{\partial x} g(x) + f(x) \frac{\partial g(x)}{\partial x},
\end{eqnarray}
得到
\begin{eqnarray}
    \langle p(t) \rangle = m \intinf x \left(
        \frac{\partial \psi^*}{\partial t}\psi + \psi^* \frac{\partial \psi}{\partial t}
    \right)
    \dd t,
    \label{eq:pt_ave_2}
\end{eqnarray}
按照自由粒子的薛定谔方程 \eqref{eq:free_partial_schrodinger}, 上式 \eqref{eq:pt_ave_2} 有
\begin{align}
    & \frac{\partial \psi^*}{\partial t} = - \frac{\ii \hbar}{2m} \frac{\partial^2 \psi}{\partial x^2}, \\
    & \frac{\partial \psi}{\partial t} = \frac{\ii \hbar}{2m} \frac{\partial^2 \psi}{\partial x^2}, 
\end{align}
代回 \eqref{eq:pt_ave_2} 有
\begin{eqnarray}
    \langle p(t) \rangle = -\frac{\ii\hbar}{2} \intinf x \left( \psi \frac{\partial^2\psi^*}{\partial x^2} - \psi^* \frac{\partial^2\psi}{\partial x^2}\right) \dd x
    \label{eq:pt_ave_3}
\end{eqnarray}
约掉了质量 $m$。再次利用复合函数求导,设 $f(x) = \psi^*(x,t)$,$g(x) = \frac{\partial^2 \psi}{\partial x}$,则
\begin{align}
\frac{\partial}{\partial x}(f g) = \frac{\partial}{\partial x}\left(\psi^* \frac{\partial^2 \psi}{\partial x^2}\right) = \psi^* \frac{\partial^2\psi}{\partial x^2} + \frac{\partial \psi^*}{\partial x} \frac{\partial \psi}{\partial x}, 
\end{align}
于是有
\begin{align}
    &\psi^* \frac{\partial^2 \psi}{\partial x^2} = \frac{\partial}{\partial x} \left(\psi^*\frac{\partial \psi}{\partial x}\right) -\frac{\partial \psi^*}{\partial x} \frac{\partial \psi}{\partial x}, \\
    &\psi \frac{\partial^2 \psi^*}{\partial x^2} = \frac{\partial}{\partial x} \left(\psi\frac{\partial \psi^*}{\partial x}\right) - \frac{\partial \psi}{\partial x} \frac{\partial \psi^*}{\partial x},
\end{align}
代回 \eqref{eq:pt_ave_3} 有
\begin{eqnarray}
    \langle p(t) \rangle = \frac{\ii\hbar}2 \intinf x \frac{\partial}{\partial x} \left(\psi^* \frac{\partial \psi}{\partial x} - \psi \frac{\partial \psi^*}{\partial x}\right) \dd x,
    \label{eq:pt_ave_4}
\end{eqnarray}
设 $f(x) = \psi^* \frac{\partial \psi}{\partial x} - \psi \frac{\partial \psi^*}{\partial x}$,数学上显然有
\begin{eqnarray}
    \frac{\partial}{\partial x}[x f(x)] = f(x) + x \frac{\partial f(x)}{\partial x},
\end{eqnarray}
所以 \eqref{eq:pt_ave_4} 为
\begin{eqnarray}
    \langle p(t) \rangle = \frac{\ii\hbar}2 \intinf \frac{\partial}{\partial x}(x f(x)) \dd x - \frac{\ii\hbar}2 \intinf f(x) \dd x,
    \label{eq:pt_ave_5}
\end{eqnarray}
其中第一项为 $x f(x) |^{+\infty}_{-\infty}$。

\suppInfo{波函数的有限性}{
对于实际的原子分子体系,波函数必须在无穷远处趋于 0。比如坐标的平均值
\begin{eqnarray}
    \langle x(t) \rangle = \intinf \psi^*(x,t) \ x\  \psi(x,t) \dd x = \intinf x \rho(x,t) \dd x
\end{eqnarray}
是偶极,必然为有限值 $\psi^* x \psi |_{x \rightarrow \pm \infty} = 0$。同理,$\langle x^2 \rangle$为电四极,也必须为有限值。总之,波函数必须在无穷远处为 0。那么 $ \psi^*\psi|_{x=\pm\infty} \propto 1/x^s, s>2 $ ,应当比$ 1/x^n $更快衰减。}

于是 \eqref{eq:pt_ave_5} 为
\begin{eqnarray}
    \langle p(t) \rangle = - \frac{\ii\hbar}2 \intinf \left(\psi^* \frac{\partial \psi}{\partial x} - \psi \frac{\partial \psi^*}{\partial x}\right) \dd x,
\end{eqnarray}
再利用分部积分,有
\begin{align}
    \langle p(t) \rangle &= \frac{\ii\hbar}2 \intinf \left(\frac{\partial}{\partial x}(\psi\psi^*) - \frac{\partial \psi}{\partial x}\psi^* - \psi^* \frac{\partial \psi}{\partial x}\right) \dd x \\
    &= \psi\psi^* |_{-\infty}^{\infty} + \frac{\ii\hbar} 2 \intinf \left(-2\psi^* \frac{\partial \psi} {\partial x} \right)\dd x \\
    &= \intinf \psi^* \left(-\ii\hbar \frac{\partial}{\partial x}\right) \dd x,
\end{align}
那么 $\hat p = - \ii\hbar \frac{\partial}{\partial x}$。
通过动量的定义、薛定谔方程,便可演绎出动量算符。

动能算符
\begin{eqnarray}
    \hat T = \frac{\hat p^2}{2m} = -\frac{\hbar^2}{2m} \frac{\partial^2}{\partial x^2}
\end{eqnarray}
即薛定谔方程左边。

\section{含势函数的 Schr\"odinger 方程表达式}%与波函数空间-时间分离}
% credit: ZY Zhu

经过上述的演绎,可以知道非相对论自由粒子的薛定谔方程为
\begin{equation}
\ii \hbar \frac{\partial \psi}{\partial t} = \frac{\hat p^2}{2 m} \psi,
\end{equation}
其中的 $\hat p^2 / 2m$ 与经典力学中的动能项 $E = p^2 / 2m$ 非常相似。而经典力学中,哈密顿量可以写为动能项加势函数项。

类似地在上述方程中引入势函数 $V(x, t)$:
\begin{equation}
\ii \hbar \frac{\partial \psi}{\partial t} = \hat H \psi = \left( \frac{\hat p^2}{2 m} + V(x, t) \right) \psi
\end{equation}
上式就是含有势函数的、一般的 Schr\"odinger 方程,同时也定义了哈密顿算符 $\hat H$。它不只可以解释自由粒子,也可以解释受限粒子 (譬如一维势箱),但无法拓展到相对论的情形。化学中通常所使用的量子力学的最根源的表达式即上述 Schr\"odinger 方程。

\subsection{定态薛定谔方程}
特别地,讨论一种常见且特殊的情况,即势函数 $V(x, t) = V(x)$ 不随时间变化。势函数不随时间变化的例子,如自由的或外加静电场的分子电子运动过程,而势函数随时间变化的例子,如分子受光激发的过程。

当 $V(x, t) = V(x)$ 时,我们说该方程存在分离变量的 (空间-时间分离的) 波函数解。假定
\begin{equation}
\psi(x, t) = \phi(x) f(t),
\end{equation}
代入到 Schr\"odinger 方程,得到
\begin{equation}
\ii \hbar \phi(x) \frac{\partial f(t)}{\partial t} = - \frac{\hbar^2}{2 m} \frac{\partial^2 \phi(x)}{\partial x^2} f(t) + V \phi(x) f(t)
\end{equation}
两边分别除以 $\psi = \phi f$,得到
\begin{equation}
\frac{\ii \hbar}{f(t)} \frac{\partial f(t)}{\partial t} = - \frac{\hbar^2}{2 m} \phi(x) \frac{\partial^2 \phi(x)}{\partial x^2} + V 
%= \frac{\hat H \phi}{\phi}
\end{equation}
该式左右分别是只关于 $t$、$x$ 的函数。因此,若要让等式成立,则等式两边必为常数。该常数我们定作 $E$,因为哈密顿算符相当于表示体系的能量 (作为守恒量),
\begin{align}
    &\hat H \phi(x) = E \phi(x), \\
    &\pdv{t} f(t) = -\frac{\ii E}\hbar f(t),
\end{align}

那么,时间部分解得
\begin{equation}
f (t) = \exp \left( - \frac{i}{\hbar} E t \right),
\end{equation}
空间部分解得
\begin{equation}
\hat H \phi(x) \equiv \qty(- \frac{\hbar^2}{2 m} \pdv[2]{x} + V (x)) \phi(x) = E \phi(x)
\end{equation}
称为\boldtext{定态} Schr\"odinger 方程。

\extraInfo{推导}{
这两节课我一直在重复物化 A I 前几节课的内容,但是这里用一些比较数学演绎的方式,推导出方程的建立过程。

我希望大家能够体会其中的创造性,这个创造性是超越于一般的演绎和推理。现在经常讲机器学习、人工智能,人工智能为什么被认为说是有前途的一门学问呢?就是说它希望超越一般的规律,去寻找复杂大数据背后的一种数据到结果的映射关系。这个映射关系如果找到足够好,是不是可以从中得到一些物理?

对于物理、化学、生物,是研究物质的科学。为什么说化学叫中心学科呢?因为它下面是可以承接物理,往上可以承接更复杂的生物。「中心学科」只个是一个中性词,并不代表它最重要。整个过程三位一体,所有理论都是要以实验为基础开展科研工作。从实验中测得一种理论,同时要反过来解释甚至指导实验。

我们现在推导的薛定谔方程本身就是从实验中间抽取出来的,这是非常典型的范例。同时,后面也通过很多的推导去告诉大家,这里蕴含理论到底是有多强大,进而可以解释多少实验。这就是量子化学方向常常要做的事情,即解释化学中的现象。
}

不失一般性的将$ \hat p, \hat T, \hat V $从一维拓展到三维,则
\begin{align}
\hat p &= -\ii\hbar \left(\pdv{x}\vb{i} + \pdv{y}\vb{j} + \pdv{z}\vb{k}\right) = -\ii\hbar \vec\nabla, \\
\hat T &= -\frac{\hbar^2}{2m}\nabla^2, \\
\hat V &= V(x,y,z)
\end{align}
三维薛定谔方程 (3D SE)
\begin{equation}
\hat H \psi %= - \frac{\hbar^2}{2 m} \left( \frac{\partial^2 \phi}{\partial x^2} + \frac{\partial^2 \phi}{\partial y^2} + \frac{\partial^2 \phi}{\partial z^2} \right) + V (x, y, z) \phi 
\equiv - \frac{\hbar^2}{2 m} \nabla^2 \psi + V (x, y, z) \psi = E \psi,
\end{equation}
其中波函数可分离变量,写成\boldtext{定态} stationary 波函数与时间分量的乘积,
\begin{equation}
\psi(x,t) = \phi(x) \exp \qty( -\dfrac{i}{\hbar} Et).
\end{equation}
%这也是在物理化学 A I 课程中讨论得最多的方程,也是这门课程中最为重要的方程。
上述是单粒子的情形,如果是多粒子,可推广至
\begin{equation}
\qty( -\sum_i \dfrac{\hbar^2}{2m_i}\nabla_i^2 + V(x_1, \cdots, x_n)) \psi(x_1,\cdots,x_n) = E \psi(x_1,\cdots,x_n)
\end{equation}

% How. 这样我们已经可以把体系的确定了方程。
% 方程全部就给出来了对吧,因为它的这个试函数有可能是时间的关方程,所以它的哈密的方程有可能时间的方程。
% 特别的,如果我们的示函数与时间无关,那么我们就可以推出哈密顿的双层的话,也与时间无关,就等于动量上浮加上 VX 对吧。那么我们就可以翻成带带着。
% 好在这方向我们要再做一次,你们物化 AE 里面已经学过了,学过的这个变量分离变量了对吧,左边跟时间无关,右边跟时间有跟坐标无关对吧算幅我说的算幅,那么我就不失一般性的,可以把不函数写成什么,时间跟空间的层级吗?
% 对吧?那带入上时左边跟时间无关,我就可以把时间提前当成常量。
% 右边跟空间无关。
% 好了,现在我们做这个事情,下一步就要把左边的右边分开,左边右边分开,就是要让左边的 FT 消耗掉,右边的 yx 消耗掉。那这个两边同时重乘向重是除向整体的波函数。对吧。
% 这是 tree 那么就有,就可以导出这个的话就是。 Up np.
% 由此我们已经把坐标跟时间分开成两边,两边是无关的两个量,它们两个要相等,就意味着这两个要同时等于一个常数,否则的话他们是没法做这样的一个连力的。那么背后的意思就是我要把它分成两个方程了。
% 对吧,这两个方程好了,对于方程。
% 对方乘 2 我们知道说这样的一个形式,我们就可以 push 一般性的设置的 FT 为 A 乘上 ESP 的 at 对吧。那么把这个带进去就变成是其实这个也不用 A 到时候再说规划。就是 AFT 等于负I。
% A 等于负的 IE 除上H8,所以 FT 就等于。
% 那么方程1。
% 就是它的叫做定态确定和方式,因为它跟时间补完了对吧。
% 当然由此我们其实已经给出体系的再说一遍,我们现在其实这两节课我一直在重复,就是像固化 A1 里面很可能才第一节或者第二节的课对吧。但是这边我们用一些比较数学的一个演绎的方式,需要把那个早期这样的一个确定的方程的推导,这个过程的话展开来了解一下。
% 我希望大家能够体会细品中间的就是一种创造性,这个创造性的话是超越于一般的这种演绎跟推理。其实包括现在经常讲继续学习人工智能。人工智能为什么被认为说其实是挺有前途的一门学问呢?就是说他其实希望就是超越一般的规律,去寻找什么复杂的大数据背后的一个纷繁的数据背后的一种什么映射关系数据到结果的映射关系。那这个映射关系中如果找到足够好,是不是可以从中认得到一些物理?这个是蛮有意思的一个情况。因为我们对于我们物理化学是物理数学都是物理化学生物,我们称之为什么学科物质科学,物质科学可能是研究物质为主的一门科学。那我们为什么说化学叫中心学科呢?因为它是承接。
% 其实这个是一个中性词,并不代表它最重要,而正是因为说它是承接了什么物理的以及下面是可以承接物理,晚上是承接,晚上是可以承接更为复杂的生物对吧?那么整个过程那三三位一体其实背后对应就是所有东西都是要以所有都要以什么实验为基础开展的科研工作对吧?那么实验从实验中测一种理论的理论,同时要反过来的话解释甚至指导实验。那我们现在的确定和方程本身就是从从实验中间抽取出来的理论的一个非常典型的范例。然后同时我们后面也通过很多的证明,很多的那个推导去告诉大家这个这样的一个范例的话,推导得到的理论到底是有多强大,到底是可以真的解释多少实验?那么这就是后面我们做的量子化学里面常常要做的事情,就解释化学中的一些现象。那么有了定态学定要方程,我们这么还要再做一步对吧。因为真实的世界是什么?三维的世界首先它不是一维的。那么我们当然要不是一般性的干嘛把这些算图从一维的干嘛变成三维的,那这个是 strike forward 对吧。
% 那动量上浮。
% Full ih bar. 我们就要对它。 Exo. rising sandwich go to panda 那么我们就引入一个新的上浮的模式。
% 定态的,我们现在后面这节课,这次这门课我们根本都碰不到磐石的,当然下节课除外,所以我们这边就做这样的一个简单的定义。由此我们已经有了这个设定二方程的三维的普遍的普适的形式,动能算服加势能算服。
% 对。这边就是从这个应该就是单例子体系对吧,他的这个定态确定和方程当地词体系在真实世界中的定态确定和方程。那么由此我们其实已经可以把确定和方程的话。
% Particular culture in.
% 之所以称为定态训练方程,就是因为它的波函数并不包含时间。
% Stationery stay.
% 有了这些我们就可以开始讲这个前面我们就到这里为止,我们其实是把算幅的一些演绎给大家了。往下我们就要引入什么?为了完整性,一门课还是要完整性,可能大家已经学过了,那我们就是还要把算法的定义重新的抽题出来描述一下,以及它的一些基本的规则。
% 算幅的定义。算幅定义从本质上来讲的话,就是干嘛把一个函数干嘛变成了另外一个函数对吧,就它是把一个函数变另外还有一个操作。那么所以原则上说 A 可以是任何一个东西,可以是标量,可以是函数,可以是泛函。泛函的意思就是说它可以是根号偏微分之类的到动量动能上浮就是一个偏微分对吧?那么我们这边就有了到底它的代数规则。
% 上浮的和与差,就是说设。
% 那么这个算幅如果 C 等于 A 加 B 那么它做 C 作用在任何一个函数上,就是等于 A 作用的函数上加上 B 中文函数上,这个是不言而喻的对吧,也不需要说任何的。也就是说 C 无论是任何的一个函数算幅,它都满足这个。 Mm.
% 算符的乘法。设 C 等于 A 乘上 B 那么 C 就用在函数上,就等于把 A 跟 B 依次作用的函数先是 B 作用的函数上,然后 A 作用在 B 作用过之后的函数。
% 那么一般情况。
% 交换率是不成立的对吧?一般情况是这样,我们什么时候是这样的?如果这个函数比如说是一个坐标算幅跟势能算幅,它肯定是可以的对吧是吧?那么我们经常会讲的就是说如果我们的算幅一个是偏微分的算幅, B 的话就是一个坐标算幅。那么。
% 就在任何一个函数上,它是什么?它应该是这个 FX 加上 X 对吧?那它其实是等于什么呢? In jiangxi.
% 那另外一边的话,如果是 xa 先作用到这个函数上,那么它就是。
% 对吧?所以它们并不相等这个。当然我相信你们学过物化 A 一定知道这个乘法,它们交换率不成立。其实本身的话就对于说微观世界里面很多新颖的这种这样的现象对吧,比如说他的不完备性,他的这个他的测不准,甚至于什么后面的还有零点能这些东西都跟这相关,这后面慢慢会展开来讲。
% 等价上浮。若 A 作用到 F3 等于 B 作用的 F3 对于任意。
% 对任意函数都成立。
% 我们就说 A 等就是等于 B 好吧,那就像这边我们说的上面这个式子的话,这个是我们对于任一个函数做下来的对吧,FX我们没有定 FX 是什么东西好,所以这边就有就知道说这个。
% 就带 xm 等于什么这两个上头对吧,是等价的。
% 当我们在定义这个算幅空间的时候,为了完备性起见,我们肯定要定义它的什么单位算法称为单位算法,据说 1 就是它的单位算法。
% Unit operator tong shi ling weikong sanfu.
% Now operate. 同时我们这边做个约定,对于常数的算法。
% 我们都不加这个帽子。那么我们从前面也可以看到说这个。
% 就等于零,就是一个空胀符。好,那么上浮的最后的定义的话,行吧,那我们下节课再讲好。
% 作业的话,我会在下节课布置。
% 哪一个?就是那个减 1 等于就是那个半 X 那个减 1 等于0。
% 没就展示他们两个算不加价是完整,这只是为了完备性。其实后面可能到时候在后面会用到一些,但是本质上的话就是表示说这两个上浮是等价的,他们相应点基本是什么都没有,就是零售这是上浮空间。为了完备,你现在我们定义放射空间,我们只会定义说我们的那这块是要空间,我们定义的空间一定要给他完整全是数学项的一个,就是事实上我们用的不多。Ok。
% 然后之前那个就是那就是猜猜是分离的这是,时空分离什么地方不明白。
% 我说我也有问题。
% 行,大概就是就是从就是这里的话,把它变量分离之后。

\chat{%
到目前为止,我们已经演绎得到一些常见的算符。为了这门课的完整性,还要把算符的定义和基本规则重新抽提出来描述一下。
}

\section{算符的定义}
算符从本质上来讲,是把一个函数变成另外一个函数的操作。
\chat{%
原则上说算符 $\hat A$ 可以是任何东西,可以是标量、函数、泛函等。如果算符是泛函,是说它可以是根号、偏微分之类的操作,比如动量算符就是偏微分。
}
\subsection{代数规则}

\begin{theorem}[算符的和与差]
    定义 $\CC = \AA + \BB$,那么
\begin{eqnarray}
    \CC u = \AA u + \BB u. 
\end{eqnarray}
\end{theorem}

\begin{theorem}[算符的乘法]
    设 $\CC = \AA \BB$,那么
\begin{eqnarray}
    \CC u = \AA\BB u = \AA (\BB u)
\end{eqnarray}
\end{theorem}

\chat{%
相信同学们学过物化一定知道这个乘法的交换律不成立,微观世界有很多新颖的现象,比如不完备性、测不准,甚至于后面会讲到的零点能等。
}

比如 $\AA = \frac{\partial}{\partial x}, \BB=x$, 有
\begin{align}
    \hat A \hat B f(x) &= \pdv{x} [x f(x)] = \qty(1 + \hat x \pdv{x}) f(x) \\
    \hat B \hat A f(x) &= x \pdv{x} f(x)
\end{align}

\begin{theorem}[等价算符]
    若 $\AA f=\BB f$ 对任意函数均成立,那么 $\AA =\BB$。
\end{theorem}
如
\begin{eqnarray}
    \frac{\partial}{\partial x} = 1 + \frac{\partial}{\partial x}.
\end{eqnarray}

\begin{theorem}[基本算符]
    单位算符 unit operator $\hat 1 = 1$,空算符 null operator $\hat 0 = 0$,常数算符通常不加 $\hat\cdot$。
\end{theorem}

% 
% 上午回顾了量子力学的诞生。
% 通过波粒二象性,隐入了实物波的波动方程,演绎出来薛定谔方程。
% 由此确立了微观粒子的状态也可以用波动方程描述。
% 对于波函数,状态与可观测量的关系,需要将算符作用上去,这是物化一的课程。
% 早上第二个内容就是演绎了动量算符为什么是偏微分的形式,并由一维拓展到高维,给出了定态薛定谔方程。
% 后面量化的大部分时间都会围绕定态薛定谔方程的求解,从定态薛定谔放成功可以得到物理化学光谱的信息。

% 第二部分,看一下算符的数学规则。
% 上午看了和差的定义是显而易见的,算符作用在函数上等于另一个函数,所以算符是定域的。$A$ 作用上去和 $B$ 作用上去是独立的。但是乘法不一样,$\hat A\hat B$ 依次作用上去。

% 一个典型的例子是导数算符和位置算符,导数算符可类比于之前推导的动量算符。

在其它表象下,算符本身可以构成完备集,算符本身可以描述微观状态,因此算符有各种加减乘除、单位元的定义,自然而然还有逆算符的定义。

\begin{theorem}[逆算符]
    设 $\hat A\hat B f(x) = f(x)$,我们就说 $\hat A = \hat B^{-1}$ 或 $\hat B = \hat A^{-1}$。
\end{theorem}

因此引申出很重要的概念\boldtext{对易}。

\begin{theorem}[对易]
    \begin{equation}
        [\hat A, \hat B] \equiv \hat A \hat B -\hat B \hat A
    \end{equation}
    若 $\hat A \hat B = \hat B \hat A$ 则$[\hat A,\hat B] = 0$,则称算符 $\hat A, \hat B$ 对易,否则不对易。
\end{theorem}

\chat{%
简单的例子,$\hat A = \frac{\partial}{\partial x}, \hat B = 3$,则二者对易。若 $\hat A = \frac{\partial}{\partial x}, \hat B = \hat x$, 则 $[\frac{\partial}{\partial x},\hat x] = 1$。
}

\homework{
    \textbf{1.} ~ 请简化下列定义的算符 $\hat B$
    \begin{equation}
        \hat A = \frac{\partial}{\partial x} x, 
        \hat B = \hat A^2
    \end{equation}
    计算对易子 
    \begin{equation}
        \left[x^3, \frac{\dd}{\dd x}\right], \left[\frac{\dd}{\dd x}, 5x^2 + 3x + 4\right]. 
    \end{equation}
}
交换不成立,于是有了对易子的规律。以上是对于最普适算符给出的定义。

\subsection{线性算符}

对于量子力学中的算符,引入约定\boldtext{线性算符},指分别作用在两个函数上,
\begin{equation}
    \hat A [f(x) + g(x)] = \hat A f(x) + \hat A g(x)
\end{equation}
这个与加和有什么不同?加和是对所有算符成立,但是上式仅对线性算符成立。线性算符是说,一个算符作用在两个函数上,相当于分别作用在两个函数上。

引入线性算符的重要性:\textbf{量子力学中所有具有物理意义的算符都是线性算符}。有一个重要的性质
\begin{equation}
    \hat A [C f(x)] = C\hat A f(x)
\end{equation}
\begin{equation}
    \hat A[ \underbrace{f(x) + \cdots + f(x)}_{C} ] = C\hat A f(x)
\end{equation}
哪些不是?$\sqrt{\cdot}$、$(\cdot)^2$、$\sin$、$\cos$ 等,比如
\begin{equation}
    \sqrt{f(x) + g(x)} \neq \sqrt{f(x)} + \sqrt{g(x)}. 
\end{equation}

\chat{%
哥本哈根诠释,表示波函数的平方表示概率。波函数本身没有物理意义,算符作用上去才有物理意义。
}

如果 $\hat A$ 是本征态,那么必然有
\begin{eqnarray}
    \hat A f(x) = A f(x), \\
    \hat A C f(x) = AC f(x)
\end{eqnarray}
任何的缩放也都是本征态,如果是非线性算符则无法满足。
\chat{%
对于波函数而言,最重要的概念是叠加,干涉、衍射,是线性加和的,如果波函数不是线性的,那么波函数就不满足后面要学的量子力学公设。
}

有了这样的线性算符,就可以给出算符的\boldtext{左右分配律},
\begin{eqnarray}
    (\hat B + \hat C) \hat A = \hat B \hat A + \hat C \hat A,
\end{eqnarray}
左右分配律对任何算符一定成立,但是右分配律仅对线性算符成立。
\begin{eqnarray}
    \hat A (\hat B + \hat C)  \neq \hat A \hat B + \hat A \hat C,
\end{eqnarray}
\chat{%
算符在量化中用得不多,如果以后做实验,实验中越来越向微观调控发展。当你去操控分子,你的操作就对应于算符,需要先设计算符,算符必须是微观可观测的。可以做一些先验的设计,否则实验是不可能成立的。
}
\homework{\textbf{2.} ~ 证明算符的左右分配律}

\subsection{Hermitian 算符与量子力学公设}

\begin{theorem}[Hermite 算符]
    对于任意给定的(合理)波函数 $u$,若算符满足
\begin{eqnarray}
    \int u^* \hat A u \dd \tau = \int(\hat A u)^* u \dd t,
\end{eqnarray}
则 $\hat A$ 为Hermite 算符。
\end{theorem}

如动量 $\hat p = -\ii \hbar \frac{\partial}{\partial x}$、动能 $\hat T = \frac{\hat p^2}{2m}$、哈密顿算符 $\hat H$。
\homework{\textbf{3.} ~ 证明上述三个算符为 Hermite 算符}

为什么我们要隐入 Hermite 算符的定义?
清晰的客观物理事实,所有的物理可观测量都对应 Hermite 算符,期望值一定是实数。对应着
\begin{align}
    &\left\langle A\right\rangle = \int \psi^* \hat A \psi \dd \tau, \\ 
    &\left\langle A\right\rangle^* = \int \psi (\hat A \psi)^* \dd \tau = \int (\hat A \psi)^* \psi \dd \tau,
\end{align}
若 $\hat A$ 非 Hermite 则
\begin{eqnarray}
    \left\langle A\right\rangle \neq \ev{A}^*
\end{eqnarray}
$\ev{A}$ 不是实数,进而 $\hat A$ 不对应物理可观测量。

我们讨论的大部分算符不只是 Hermite 的,也都是线性的。
\homework{\textbf{4.} ~ 对于线性的 Hermite 算符 $\hat A$,我们有
\begin{eqnarray}
    \int u^* \hat A \nu \,\dd \tau = \int (\hat A u)^* \nu \,\dd \tau,
\end{eqnarray}
同时,对于线性的 Hermite 算符,$\hat A, \hat B$ 具有以下性质,
\begin{enumerate}
    \item $\hat A + \hat B$ 仍是 Hermite 算符
    \item $[\hat A, \hat B] = 0$,$\hat A\hat B$ 与 $\hat B \hat A$ 仍是 Hermite 算符
\end{enumerate}

\textbf{5.}
    若 $\hat A, \hat B, \hat C$ 为线性算符
    \begin{enumerate}
        \item 若 $\hat A, \hat B$ 同时为 Hermite 算符,则 $\hat A \hat B + \hat B \hat A$ 与 $\ii [\hat A \hat B - \hat B \hat A]$ 都是 Hermite 算符
        \item $[\hat A, \hat B \hat C] = \hat B [\hat A, \hat C] + [\hat A, \hat B]\hat C$
        \item $[\hat A \hat B, \hat C] = \AA[\BB,\CC] + [\AA,\CC]\BB$
        \item $[\AA,[\BB,\CC]] + [\BB,[\CC,\AA]] + [\CC, [\AA, \BB]] = 0$
    \end{enumerate}
}

Hermite 算符之后,还有个重要的约定。

\subsection{Dirac 记号}

它是用来描述期望值。如果算符作用在波函数上,
\begin{eqnarray}
    \int \phi_m^* \hat A \phi_n \dd\tau = \langle \phi_m | \hat A | \phi_m \rangle \equiv \langle m | \hat A | m \rangle \equiv A_{mn}
\end{eqnarray}
如果 $m = n$ 则表示期望值,如果不等于则表示跃迁矩阵元,
\begin{eqnarray}
    A_{mn} \equiv \langle m | \hat A | n \rangle
\end{eqnarray}
% $\langle \phi_m | \hat A | \phi_m \rangle$ Dirac 记号,$A_{mn}$ 矩阵元表述,$\langle m | n \rangle = \int \phi_m^* \phi_n \dd\tau$ 重叠积分。所以
左矢表示 $\langle m| = \phi_m^*$ 波函数的共轭,右矢表示 $\ket{m} = \phi_m$ 波函数。还有更多自由的写法,
$\hat A n$ 自身可以看成一个函数,
\begin{eqnarray}
    \langle m | \hat A | n \rangle = \langle m | \hat A n \rangle,
\end{eqnarray}

所以 Dirac 记号右矢 $|n\rangle$ 代表波函数 $\phi_n$(即 $\hat A \phi_n$),左矢 $\langle m |$ 代表波函数 $\phi_m$(即 $\hat A \phi_m$)的共轭。

% \begin{eqnarray}
%     \int u^* \hat A \nu \dd\tau = \int(\hat Au)^* \nu \dd\tau, \\
%     \langle u | \hat A | u \rangle = \langle \hat A u | u \rangle = [][][][]
% \end{eqnarray}

通过 Dirac 记号,可以写出 Hermite 算符的定义:
\begin{equation}
  \langle m | \hat A n \rangle = \langle \hat A m | n \rangle = \langle n | \hat A m \rangle^* = \langle m | \hat A^* n \rangle
\end{equation}
对于 Hermite 算符而言,$\hat A = \hat A^*$。

但需要留意的是,当 $\hat A = \partial_x$ 即导数算符时,$\hat A^* = - \partial_x$。因此算符的共轭并非简单地将虚数单位 $\ii$ 替换为 $-\ii$。

\chat{%
我们把量子力学里算符简单过一遍。假设物化一里都学过,为了课程完整性所以需要讲一遍。这里跟以前学的有何不同?如果没有问题,下节课会讨论量子力学公设。如果有时间,我们会讨论物理意义和一些具体例子。
}

实验归纳总结了许多公理体系,即量子力学公设。目前还没有遇到违反公设的情况。多种公设的表述方式,在数学证明中都是等价的。

\begin{theorem}[公设1]
    一个量子力学体系,均可用一个含时的波函数 $\psi(\vec r, t)$ 来描述,该函数应是单值、连续和有限的品优波函数。
\end{theorem}

\begin{theorem}[公设2]
对于一个量子体系,每一个可观测力学量都对应一个线性 Hermite 算符与量子力学公设
\end{theorem}

\begin{theorem}[公设3]
当对量子体系的某一力学量进行测量时,每次可测一数值 $\lambda$ 则 $\lambda$ 与该力学量对应的算符以及波函数 $\psi$ 之间存在如下关系
\begin{eqnarray}
    \hat F \psi =\lambda \psi
\end{eqnarray}
即 $\lambda$ 是 $\hat F$ 的本征值,而 $\psi$ 是 $\hat F$ 的本征函数。
\end{theorem}

\chat{
以前有同学问:为什么实验中的基态在测完之后还是在基态?化学中的实验,为什么能提前知道物质处在基态,为什么没有不确定性?比如为什么物质的生成焓都是确定的?我觉得这个问题非常好。
}

\begin{theorem}[公设4]
对于任意一个线性 Hermite 算符,其本征方程
\begin{eqnarray}
    \hat F \psi_i = \lambda_i \psi_i
\end{eqnarray}
的本征函数 $\{\psi_i\}$ 均构成一个完备集,对于任意一个波函数,
\begin{eqnarray}
    \Psi = \sum_i c_i \psi_i
\end{eqnarray}
都可以可以展开。
\end{theorem}

这是一个非常强的定理,有它物理意义。
\chat{%
一个原子的哈密顿算符,我们在物化1中解出了各种轨道。按第四点来说,这个完备基是在全空间展开的,也就是任何体系都可以用氢原子的完备基展开。比如两个相距一定距离的氢原子,物理上来说用两个波函数叠加是最好的,但从第四点来说,用一个原子的波函数来展开也是完全可行的。实际上,因为波函数是无穷多的,必须用无穷多才能完全描述,因此可以这么描述,但不是最高效的。
}

\subsection{Hermite 算符}
给出 Hermite 算符的特殊性质。不推导,留一些给同学们证明。

\begin{theorem}[a]
    线性 Hermite 算符的本征函数 $\{\psi_i\}$ 一定是正交归一的,Dirac 记号表述为
\begin{align}
    &\langle i | i \rangle = 1, \\
    &\langle i | j \rangle = 0, \quad i \neq j
\end{align}
推知
\begin{eqnarray}
    \delta_{ij} = 
    \left\{
        \begin{matrix}
            1, i=j,\\
            0, i\neq j
        \end{matrix}
    \right.,
\end{eqnarray}
\end{theorem}

\begin{theorem}[b]
    若$\psi$ 是算符 $\hat F$ 的本征值 $f$ 的本征函数,
\begin{eqnarray}
    \hat F \psi = f \psi,
\end{eqnarray}
则对 $\psi$ 体系,$\hat F$ 的一次测量肯定($100\%$)测得 $f$ 值。
\end{theorem}

\begin{theorem}[c]
若 $\Psi$ 不是 $F$ 的本征函数,按照第四点完备基的定义,总是可以表述成
\begin{eqnarray}
    \Psi = \sum_i c_i \psi_i, \quad\langle \Psi | \Psi \rangle = 1
\end{eqnarray}
如果是归一的,利用正交归一的性质
\begin{align}
    \langle \Psi | \Psi \rangle &= \Big\langle \sum_i c_i \psi_i \Big| \sum_j c_j \psi_j \Big\rangle \\
    &= \sum_i \sum_j c_i^* c_j \langle \psi_i | \psi_j \rangle \\
    &= \sum_i \sum_j c_i^* c_j \delta_{ij} \\
    &= \sum_i |c_i|^2 = 1.
\end{align}
\end{theorem}
算符的期望值
\begin{eqnarray}
    \langle F\rangle = \langle \Psi | \hat F | \Psi \rangle = \sum_i |c_i|^2 f_i. 
\end{eqnarray}
如果波函数并不是算符 $\hat F$ 的本征函数,每进行一次测量,平均值即为上式,每次测得的值为 $f_i$,概率为 $c_i$。

\chat{%
薛定谔猫是处在生态和死态叠加中,如果猫的状态是生和死的叠加,它并不是一定生或死的。那么数学上来说,我对某状态做某测量得到某值的概率是多少。

我们知道,动量算符和位置算符对易,不能同时测量,换句话说,它们不能同时拥有相同的本征函数。如果对动量的本征函数做位置的测量,也就是把位置的算符作用到动量的本征函数上去。
假设所有东西都是离散的。设 $\{\phi_i\}$ 为 $\hat p$ 的本征函数,一定不是位置算符的本征函数。将坐标作用上去,平均值为 $\int \phi_i^* \hat x \phi_i \dd \tau$。位置算符 $\hat x$ 的本征函数 $\{f_m\}$,将动量的本征函数展开为 $\phi_i = \sum_m c_i f_i$,所以在某个动量的本征函数 $\phi_i$ 上找到位于 $x_m$ 电子的概率就是 $|c_m|^2$,
测量后就有 $|c_m|^2$ 概率坍缩到 $f_m$ 上去。这个时候再测量位置,那么这个体系将永远处在 $f_m$ 上了。
如果再去测量动量,重新展开成 $f_m = \sum_i c_i \phi_i$,动量是全空间覆盖的,那么就有 $|c_i|$ 的概率处在 $\phi_i$ 态上。

我们在讨论化学问题的时候,总是在说基态能量、生成热,因为我们是对能量算符进行测量,所以基态的能量是不会变的,除非是基态和激发态的耦合或者是散射过程。
}

\begin{theorem}[d]
    若两个线性 Hermite 算符有一个共同的本征函数完备集,则两个算符可以对易,
\begin{eqnarray}
    \hat F \psi_i = f_i \psi_i. \hat G \psi_i = g_i \psi_i,
\end{eqnarray}
则二者对易 $[\hat F, \hat G] = 0$ 或二者可以同时测量
\end{theorem}

\homework{\textbf{6.} ~ 证明 (a) 与 (d)}

为什么单值、连续、有限很重要?去年讲太慢了,很多没讲到。
我们用哥本哈根的统计诠释,讲品优波函数的物理内涵。

其实波函数本身没有物理意义,波函数的平方是概率,有物理意义。
\begin{eqnarray}
    |\psi (x,t)|^2 = \psi^*(x,t) \psi(x,t)
\end{eqnarray}
为概率密度 probability density,在 $t$ 时刻 $x$ 位置到 $x + \dd x$ 为之内发现粒子的概率,
\begin{eqnarray}
    |\psi(x,t)|^2\dd x = \text{Pr}(\dd x; x; t)
\end{eqnarray}
在区间 $[a,b]$ 内发现例子的概率,
\begin{eqnarray}
    \int_a^b |\psi(x,t)|^2 \dd x = \text{Pr} (a \leqslant x \leqslant b, t)
\end{eqnarray}
全空间的概率和
\begin{eqnarray}
    \int_{-\infty}^{\infty} |\psi(x,t)|^2 \dd x = 1
\end{eqnarray}
$\psi$ 是归一化的 normalized。

由这些定义出发,品优波函数有哪些约束?

a) $|\psi|^2$ 是概率密度,可知 $|\psi(x,t)|^2 \dd x$ 有限。
\chat{%
那么这个约定,是否意味着波函数在任何一点都要有限呢?实际上并不是很重要,其背后的数学表述是波函数必须平方可积的 square-integrable,才能保证波函数可以归一。这里有例外吗?第一节课我们就碰到过,自由粒子是全空间发散,下节课会讨论到。

在量子力学中,并不排除会使用某些不能归一化的波函数。\footnote{曾谨言 \S 2.1.6; Levine \S 2.4, \S 3.8}一个重要的例外就是自由粒子。波函数有限的定义可能不是很严谨。
}

\suppInfo{波函数一定为有限值才能平方可积吗?}
{
    假设在三维空间中 $x_0 $ 处,波函数存在孤立奇点 $ |\Psi(x_0)|^2 \rightarrow \infty $。看似违反了「有限」的公设,实际上,
	只需要 $\int_{\tau_0} |\psi(x)|^2 \dd[3] x$ 为有限值,在物理上就可以接受,
	$ \tau_0 $为包围$ x_0 $附近的任意小体积。
	
	取$ x_0 = 0 $,在球坐标下展开,
	\begin{align}
	\int_{\tau_0} |\psi(x)|^2 \dd[3] x &= \int_0^\pi \int_0^{2\pi} \int_0^{r_0} |\Psi(r,\theta,\phi)|^2 r^2\dd r \sin\theta\dd\theta \dd\phi \\
	&\propto \int_0^{r_0} |\Psi(r)|^2 r^2 \dd r \\
	&\propto \frac{r^3}{3} |\Psi(r)|^2 \Big|_0^{r_0}
	\end{align}
	当$ r_0 \rightarrow 0 $时,$r_0^3 |\psi(r_0)|^2 \rightarrow 0$,设 $\psi(r_0)| \rightarrow \frac1{r_0^s}$,则 $r_0^3 \frac{1}{r_0^{2s}} \rightarrow 0$,解得 $s < \frac32$。

    对于其它维度,可解得 二维$ s < 1 $、 一维$ s < \frac{1}{2} $。
}

b) 当 $x\rightarrow\pm\infty, \psi(x,t)\rightarrow0$,积分是全空间的,并且最终是有限的,所以在无穷远处一定为0。
\chat{%
如果在无穷远处不为 0,肯定积分发散的。所以这一条也是概率密度推导出来的。
}

c) 为了满足哥本哈根学派的统计解释,波函数必须归一化,所以全空间模的积分为1
\begin{eqnarray}
    \int_{-\infty}^{\infty} |\psi(x,t)|^2 \dd x = 1,
\end{eqnarray}
\suppInfo{波函数归一化}{
但归一化并不像想象中自发成立的。

问题,如果波函数在 $t = 0$ 是归一化的 $\int_{-\infty}^{\infty} |\psi(x,0)|^2 \dd x = 1$,那么它时间演化中的每一刻 $t>0$ 是否都是归一化的?请证明
\begin{eqnarray}
    \frac{\partial}{\partial t} \int_{-\infty}^\infty |\psi(x,t)|^2 \dd x =0, \rightarrow \int_{-\infty}^\infty \frac{\partial}{\partial t} |\psi|^2 \dd x =0,
\end{eqnarray}

证明,
% \begin{eqnarray}
%     \frac{\partial}{\partial t}|\psi|^2 = \psi \frac{\partial}{\partial t} \psi^* + \psi^* \frac{\partial}{\partial t}\psi
% \end{eqnarray}
% 代入薛定谔方程,有
% \begin{align}
%     &\ii\hbar \frac{\partial}{\partial t} \Psi = -\frac{\hbar^2}{2m} \frac{\partial^2}{\partial x^2} \psi + V(x) \psi, \\
%     &\frac{\partial}{\partial t} \psi = \frac{\ii\hbar}{2m} \frac{\partial^2\psi}{\partial x^2} - \frac{\ii}{\hbar} V(x) \psi, \\
%     123
% \end{align}
% 将后两式子代回 [][][],则有 【】【很长的式子,】【】
Schr\"odinger方程具有保持波函数归一化的特性。\footnote{顾樵 \S 2.1.2}证明如下

    	随时间演化满足含时薛定谔方程
    	\begin{equation}
    	\ii\hbar \pdv{t} \Psi = -\frac{\hbar^2}{2m} \pdv[2]{x}\Psi + V\Psi
    	\end{equation}
由
    \begin{align}
    \pdv{t} \Psi &= \dfrac{\ii\hbar}{2m} \pdv[2]{x} \Psi - \dfrac{\ii}{\hbar} V(x) \Psi \\
    \pdv{t} \Psi^* &= -\dfrac{\ii\hbar}{2m} \pdv[2]{x} \Psi^* + \dfrac{\ii}{\hbar} V(x) \Psi^*
    \end{align}
得
    	\begin{align}
    	&\phantom{=}\pdv{t} |\Psi(x,t)|^2  \\
        &= \pdv{\Psi^*}{t}\Psi + \Psi^*\pdv{\Psi}{t}  \\
    	&= \Psi \qty(-\dfrac{\ii\hbar}{2m} \pdv[2]{x} \Psi + \dfrac{\ii}{\hbar} V(x) \Psi) + \Psi^* \qty(\dfrac{\ii\hbar}{2m} \pdv[2]{x} \Psi - \dfrac{\ii}{\hbar} V(x) \Psi) \\
    	&= \dfrac{\ii\hbar}{2m} \qty[\Psi^*\pdv[2]{\Psi}{x} - \Psi\pdv[2]{\Psi^*}{x}] \\
    	&= \dfrac{\ii\hbar}{2m} \pdv{x} \qty[\Psi^*\pdv{\Psi}{x} - \Psi\pdv{\Psi^*}{x}]
    	\end{align}
那么
    \begin{align}
    \pdv{t} \intinf |\Psi(x,t)|^2 \dd x &= \intinf \dfrac{\ii\hbar}{2m} \pdv{x} \qty[\Psi^*\pdv{\Psi}{x} - \Psi\pdv{\Psi^*}{x}] \dd x \\
    & = \dfrac{\ii\hbar}{2m} \qty[\Psi^*\pdv{\Psi}{x} - \Psi\pdv{\Psi^*}{x}] \Big|_{-\infty}^\infty \\
   &= 0 
    \end{align}
 所以$ \intinf |\Psi(x,t)|^2 \dd x  $不依赖于时间。证毕。
}

\chat{%
甚至可以讲到高斯定理。

下节课真正开始求解真正体系了,我们会做一些拓展,展开数学推导中的黑箱。
}

\extraInfo{助教补充}{
本周内容与物理化学 A(I) 的区别不大,只是已有知识点的深化。

引入 Dirac 记号会大大方便公式表达,尽管刚刚上手可能会辛苦一些。我们的课程中,绝大部分时
间仍然会从波函数的角度讨论问题,因此将 Dirac 记号看成函数是不会有问题的;如果有任何不容
易理解的情况,不妨将 Dirac 记号转换为波函数积分记号。

本次课多了一些在以前课程中不强调的数学技巧,譬如分部积分、波函数在无穷远处的性质、积分或求
和记号与偏导可交换等。
}


